%% Generated by Sphinx.
\def\sphinxdocclass{report}
\documentclass[letterpaper,10pt,english]{sphinxmanual}
\ifdefined\pdfpxdimen
   \let\sphinxpxdimen\pdfpxdimen\else\newdimen\sphinxpxdimen
\fi \sphinxpxdimen=.75bp\relax

\PassOptionsToPackage{warn}{textcomp}
\usepackage[utf8]{inputenc}
\ifdefined\DeclareUnicodeCharacter
% support both utf8 and utf8x syntaxes
  \ifdefined\DeclareUnicodeCharacterAsOptional
    \def\sphinxDUC#1{\DeclareUnicodeCharacter{"#1}}
  \else
    \let\sphinxDUC\DeclareUnicodeCharacter
  \fi
  \sphinxDUC{00A0}{\nobreakspace}
  \sphinxDUC{2500}{\sphinxunichar{2500}}
  \sphinxDUC{2502}{\sphinxunichar{2502}}
  \sphinxDUC{2514}{\sphinxunichar{2514}}
  \sphinxDUC{251C}{\sphinxunichar{251C}}
  \sphinxDUC{2572}{\textbackslash}
\fi
\usepackage{cmap}
\usepackage[T1]{fontenc}
\usepackage{amsmath,amssymb,amstext}
\usepackage{babel}



\usepackage{times}
\expandafter\ifx\csname T@LGR\endcsname\relax
\else
% LGR was declared as font encoding
  \substitutefont{LGR}{\rmdefault}{cmr}
  \substitutefont{LGR}{\sfdefault}{cmss}
  \substitutefont{LGR}{\ttdefault}{cmtt}
\fi
\expandafter\ifx\csname T@X2\endcsname\relax
  \expandafter\ifx\csname T@T2A\endcsname\relax
  \else
  % T2A was declared as font encoding
    \substitutefont{T2A}{\rmdefault}{cmr}
    \substitutefont{T2A}{\sfdefault}{cmss}
    \substitutefont{T2A}{\ttdefault}{cmtt}
  \fi
\else
% X2 was declared as font encoding
  \substitutefont{X2}{\rmdefault}{cmr}
  \substitutefont{X2}{\sfdefault}{cmss}
  \substitutefont{X2}{\ttdefault}{cmtt}
\fi


\usepackage[Bjarne]{fncychap}
\usepackage{sphinx}

\fvset{fontsize=\small}
\usepackage{geometry}


% Include hyperref last.
\usepackage{hyperref}
% Fix anchor placement for figures with captions.
\usepackage{hypcap}% it must be loaded after hyperref.
% Set up styles of URL: it should be placed after hyperref.
\urlstyle{same}

\addto\captionsenglish{\renewcommand{\contentsname}{JADE User Guide:}}

\usepackage{sphinxmessages}
\setcounter{tocdepth}{1}



\title{JADE}
\date{Jun 24, 2021}
\release{v1.2.0}
\author{Davide Laghi}
\newcommand{\sphinxlogo}{\vbox{}}
\renewcommand{\releasename}{Release}
\makeindex
\begin{document}

\pagestyle{empty}
\sphinxmaketitle
\pagestyle{plain}
\sphinxtableofcontents
\pagestyle{normal}
\phantomsection\label{\detokenize{index::doc}}


Version: v1.2.0

JADE is a new tool for nuclear libraries V\&V.
Brought to you by NIER, University of Bologna (UNIBO) and Fusion For Energy (F4E).

JADE is an open\sphinxhyphen{}source software licensed under the {\hyperref[\detokenize{LICENSE:gnulicense}]{\sphinxcrossref{\DUrole{std,std-ref}{GNU GPLv3 License}}}}.
When using JADE for scientific publications you are kindly encouraged to cite the following papers:
\begin{itemize}
\item {} 
Davide Laghi et al, 2020, “JADE, a new software tool for nuclear fusion data libraries verification \& validation”,
Fusion Engineering and Design, \sphinxstylestrong{161} 112075, doi: \sphinxurl{https://doi.org/10.1016/j.fusengdes.2020.112075}.

\end{itemize}

For additional information contact: \sphinxhref{mailto:d.laghi@nier.it}{d.laghi@nier.it}

For additional information on future developments please check the issues list on the
GitHub repository {[}link{]}.


\chapter{JADE in a nutshell}
\label{\detokenize{nutshell:jade-in-a-nutshell}}\label{\detokenize{nutshell::doc}}
\noindent\sphinxincludegraphics[width=600\sphinxpxdimen]{{scheme}.png}

JADE is a new tool for nuclear libraries V\&V.
Brought to you by NIER, University of Bologna (UNIBO) and Fusion For Energy (F4E).
JADE is an open source, Python 3 based software able to:
\begin{itemize}
\item {} 
automatically build a series of MCNP input file using different nuclear
data libraries;

\item {} 
automatically run simulations on such inputs;

\item {} 
automatically parse and post\sphinxhyphen{}process all the generated MCNP outputs.

\end{itemize}

The benchmarks implemented by default are divided between computational
and experimental benchmarks. The post\sphinxhyphen{}processing output includes:
\begin{itemize}
\item {} 
raw data in .csv files containing the entire tallied output from the
simulations;

\item {} 
formatted Excel recap files;

\item {} 
Word and PDF atlas collecting the plots generated during the post\sphinxhyphen{}processing.

\end{itemize}

Additional JADE features are:
\begin{itemize}
\item {} 
the possibility to implement user\sphinxhyphen{}defined benchmarks;

\item {} 
operate on the material card of an MCNP input (e.g. create material mixtures
or translate it to a different nuclear data library);

\item {} 
print a recap of the material composition of an MCNP input.

\end{itemize}

When using JADE for scientific publications you are kindly encouraged to cite the following papers:
\begin{itemize}
\item {} 
Davide Laghi et al, 2020, “JADE, a new software tool for nuclear fusion data libraries verification \& validation”,
Fusion Engineering and Design, \sphinxstylestrong{161} 112075, doi: \sphinxurl{https://doi.org/10.1016/j.fusengdes.2020.112075}.

\end{itemize}

For additional information contact: \sphinxhref{mailto:d.laghi@nier.it}{d.laghi@nier.it}

For additional information on future developments please check the issues list on the
GitHub repository {[}link{]}.


\chapter{Installation}
\label{\detokenize{usage/installation:installation}}\label{\detokenize{usage/installation:install}}\label{\detokenize{usage/installation::doc}}
The procedure to install JADE is the following:
\begin{enumerate}
\sphinxsetlistlabels{\arabic}{enumi}{enumii}{}{.}%
\item {} 
Install/update Anaconda, you can update all packages in your current environment using:

\sphinxcode{\sphinxupquote{conda update \sphinxhyphen{}\sphinxhyphen{}all}}

However, if bugs or problems are encountered, a fresh Anaconda re\sphinxhyphen{}installation may solve the issues.

\item {} 
Install additional packages. It may be necessary to activate the conda\sphinxhyphen{}forge channel. It can be done typing in an anaconda prompt shell:

\sphinxcode{\sphinxupquote{conda config \sphinxhyphen{}\sphinxhyphen{}add channels conda\sphinxhyphen{}forge}}

then use:

\sphinxcode{\sphinxupquote{conda install python\sphinxhyphen{}docx}}

The second package that is needed is numjuggler for parsing of MCNP inputs.
Unfortunately, this is not available for conda installation and pip should be used instead:

\sphinxcode{\sphinxupquote{pip install numjuggler}}

\item {} 
Extract the zip into a folder of choice (from now on \sphinxcode{\sphinxupquote{\textless{}JADE\_root\textgreater{}}});

\item {} 
Rename the folder containing the Python scripts as ‘Code’ (\sphinxcode{\sphinxupquote{\textless{}JADE\_root\textgreater{}\textbackslash{}Code}});

\item {} 
Open the global configuration file: \sphinxcode{\sphinxupquote{\textless{}JADE\_root\textgreater{}\textbackslash{}Code\textbackslash{}Configuration\textbackslash{}Config.xlsx}};
here you need to properly set the environment variables specified in the ‘MAIN Config.’ sheet (i.e. xsdir Path, and multithread options);

\item {} 
Open an anaconda prompt shell and change directory to \sphinxcode{\sphinxupquote{\textless{}JADE\_root\textgreater{}\textbackslash{}Code}}. Then type:

\sphinxcode{\sphinxupquote{python main.py}}

\item {} 
On the first usage the rest of the folders architecture is initialized.

\end{enumerate}

\begin{sphinxadmonition}{important}{Important:}
A limitator has been inserted in the code in order to test it before using JADE for production
(this will be eliminated when a proper function testing the installation will be produced).
To remove it, open \sphinxcode{\sphinxupquote{\textless{}JADE\_root\textgreater{}\textbackslash{}Code\textbackslash{}testrun.py}} and comment out line 239 and 298 while de\sphinxhyphen{}commenting line 238 and 297.
\end{sphinxadmonition}


\chapter{Folder Structure}
\label{\detokenize{usage/folders:folder-structure}}\label{\detokenize{usage/folders:folders}}\label{\detokenize{usage/folders::doc}}
The following is a scheme of the JADE folder structure:

\begin{sphinxVerbatim}[commandchars=\\\{\}]
\PYG{o}{\PYGZlt{}}\PYG{n}{JADE\PYGZus{}root}\PYG{o}{\PYGZgt{}}
    \PYG{o}{|}\PYG{o}{\PYGZhy{}}\PYG{o}{\PYGZhy{}}\PYG{o}{\PYGZhy{}}\PYG{o}{\PYGZhy{}}\PYG{o}{\PYGZhy{}}\PYG{o}{\PYGZhy{}}\PYG{o}{\PYGZhy{}}\PYG{o}{\PYGZhy{}}\PYG{o}{\PYGZhy{}} \PYG{n}{Benchmark} \PYG{n}{inputs}
    \PYG{o}{|}
    \PYG{o}{|}\PYG{o}{\PYGZhy{}}\PYG{o}{\PYGZhy{}}\PYG{o}{\PYGZhy{}}\PYG{o}{\PYGZhy{}}\PYG{o}{\PYGZhy{}}\PYG{o}{\PYGZhy{}}\PYG{o}{\PYGZhy{}}\PYG{o}{\PYGZhy{}}\PYG{o}{\PYGZhy{}} \PYG{n}{Code}
    \PYG{o}{|}            \PYG{o}{|}\PYG{o}{\PYGZhy{}}\PYG{o}{\PYGZhy{}}\PYG{o}{\PYGZhy{}}\PYG{o}{\PYGZhy{}}\PYG{o}{\PYGZhy{}} \PYG{n}{Default} \PYG{n}{Settings}
    \PYG{o}{|}            \PYG{o}{|}\PYG{o}{\PYGZhy{}}\PYG{o}{\PYGZhy{}}\PYG{o}{\PYGZhy{}}\PYG{o}{\PYGZhy{}}\PYG{o}{\PYGZhy{}} \PYG{n}{docs}
    \PYG{o}{|}            \PYG{o}{|}\PYG{o}{\PYGZhy{}}\PYG{o}{\PYGZhy{}}\PYG{o}{\PYGZhy{}}\PYG{o}{\PYGZhy{}}\PYG{o}{\PYGZhy{}} \PYG{n}{Installation} \PYG{n}{Files}
    \PYG{o}{|}            \PYG{o}{|}\PYG{o}{\PYGZhy{}}\PYG{o}{\PYGZhy{}}\PYG{o}{\PYGZhy{}}\PYG{o}{\PYGZhy{}}\PYG{o}{\PYGZhy{}} \PYG{n}{Templates}
    \PYG{o}{|}
    \PYG{o}{|}\PYG{o}{\PYGZhy{}}\PYG{o}{\PYGZhy{}}\PYG{o}{\PYGZhy{}}\PYG{o}{\PYGZhy{}}\PYG{o}{\PYGZhy{}}\PYG{o}{\PYGZhy{}}\PYG{o}{\PYGZhy{}}\PYG{o}{\PYGZhy{}}\PYG{o}{\PYGZhy{}} \PYG{n}{Configuration}
    \PYG{o}{|}                \PYG{o}{|}\PYG{o}{\PYGZhy{}}\PYG{o}{\PYGZhy{}}\PYG{o}{\PYGZhy{}}\PYG{o}{\PYGZhy{}}\PYG{o}{\PYGZhy{}}\PYG{o}{\PYGZhy{}}\PYG{o}{\PYGZhy{}}\PYG{o}{\PYGZhy{}}\PYG{o}{\PYGZhy{}}\PYG{o}{\PYGZhy{}}\PYG{o}{\PYGZhy{}} \PYG{n}{Benchmarks} \PYG{n}{Configuration}
    \PYG{o}{|}                \PYG{o}{|}\PYG{o}{\PYGZhy{}}\PYG{o}{\PYGZhy{}}\PYG{o}{\PYGZhy{}}\PYG{o}{\PYGZhy{}}\PYG{o}{\PYGZhy{}}\PYG{o}{\PYGZhy{}}\PYG{o}{\PYGZhy{}}\PYG{o}{\PYGZhy{}}\PYG{o}{\PYGZhy{}}\PYG{o}{\PYGZhy{}}\PYG{o}{\PYGZhy{}} \PYG{n}{Config}\PYG{o}{.}\PYG{n}{xlsx}
    \PYG{o}{|}
    \PYG{o}{|}\PYG{o}{\PYGZhy{}}\PYG{o}{\PYGZhy{}}\PYG{o}{\PYGZhy{}}\PYG{o}{\PYGZhy{}}\PYG{o}{\PYGZhy{}}\PYG{o}{\PYGZhy{}}\PYG{o}{\PYGZhy{}}\PYG{o}{\PYGZhy{}}\PYG{o}{\PYGZhy{}} \PYG{n}{Experimental} \PYG{n}{results}
    \PYG{o}{|}                    \PYG{o}{|}\PYG{o}{\PYGZhy{}}\PYG{o}{\PYGZhy{}}\PYG{o}{\PYGZhy{}}\PYG{o}{\PYGZhy{}}\PYG{o}{\PYGZhy{}}\PYG{o}{\PYGZhy{}}\PYG{o}{\PYGZhy{}}\PYG{o}{\PYGZhy{}}\PYG{o}{\PYGZhy{}}\PYG{o}{\PYGZhy{}}\PYG{o}{\PYGZhy{}}\PYG{o}{\PYGZhy{}} \PYG{o}{\PYGZlt{}}\PYG{n}{Benchmark} \PYG{n}{name} \PYG{l+m+mi}{1}\PYG{o}{\PYGZgt{}}
    \PYG{o}{|}                    \PYG{o}{|}\PYG{o}{\PYGZhy{}}\PYG{o}{\PYGZhy{}}\PYG{o}{\PYGZhy{}}\PYG{o}{\PYGZhy{}}\PYG{o}{\PYGZhy{}}\PYG{o}{\PYGZhy{}}\PYG{o}{\PYGZhy{}}\PYG{o}{\PYGZhy{}}\PYG{o}{\PYGZhy{}}\PYG{o}{\PYGZhy{}}\PYG{o}{\PYGZhy{}}\PYG{o}{\PYGZhy{}} \PYG{p}{[}\PYG{o}{.}\PYG{o}{.}\PYG{o}{.}\PYG{p}{]}
    \PYG{o}{|}
    \PYG{o}{|}\PYG{o}{\PYGZhy{}}\PYG{o}{\PYGZhy{}}\PYG{o}{\PYGZhy{}}\PYG{o}{\PYGZhy{}}\PYG{o}{\PYGZhy{}}\PYG{o}{\PYGZhy{}}\PYG{o}{\PYGZhy{}}\PYG{o}{\PYGZhy{}}\PYG{o}{\PYGZhy{}} \PYG{p}{[}\PYG{n}{Quality}\PYG{p}{]}
    \PYG{o}{|}
    \PYG{o}{|}\PYG{o}{\PYGZhy{}}\PYG{o}{\PYGZhy{}}\PYG{o}{\PYGZhy{}}\PYG{o}{\PYGZhy{}}\PYG{o}{\PYGZhy{}}\PYG{o}{\PYGZhy{}}\PYG{o}{\PYGZhy{}}\PYG{o}{\PYGZhy{}}\PYG{o}{\PYGZhy{}} \PYG{n}{Tests}
    \PYG{o}{|}            \PYG{o}{|}\PYG{o}{\PYGZhy{}}\PYG{o}{\PYGZhy{}}\PYG{o}{\PYGZhy{}}\PYG{o}{\PYGZhy{}}\PYG{o}{\PYGZhy{}}\PYG{o}{\PYGZhy{}}\PYG{o}{\PYGZhy{}}\PYG{o}{\PYGZhy{}}\PYG{o}{\PYGZhy{}} \PYG{n}{MCNP} \PYG{n}{simulations}
    \PYG{o}{|}            \PYG{o}{|}                 \PYG{o}{|}\PYG{o}{\PYGZhy{}}\PYG{o}{\PYGZhy{}}\PYG{o}{\PYGZhy{}}\PYG{o}{\PYGZhy{}}\PYG{o}{\PYGZhy{}}\PYG{o}{\PYGZhy{}}\PYG{o}{\PYGZhy{}}\PYG{o}{\PYGZhy{}}\PYG{o}{\PYGZhy{}}\PYG{o}{\PYGZhy{}}\PYG{o}{\PYGZhy{}} \PYG{o}{\PYGZlt{}}\PYG{n}{Lib} \PYG{n}{suffix} \PYG{l+m+mi}{1}\PYG{o}{\PYGZgt{}}
    \PYG{o}{|}            \PYG{o}{|}                 \PYG{o}{|}                  \PYG{o}{|}\PYG{o}{\PYGZhy{}}\PYG{o}{\PYGZhy{}}\PYG{o}{\PYGZhy{}}\PYG{o}{\PYGZhy{}}\PYG{o}{\PYGZhy{}}\PYG{o}{\PYGZhy{}}\PYG{o}{\PYGZhy{}}\PYG{o}{\PYGZhy{}}\PYG{o}{\PYGZhy{}}\PYG{o}{\PYGZhy{}} \PYG{o}{\PYGZlt{}}\PYG{n}{Benchmark} \PYG{n}{name} \PYG{l+m+mi}{1}\PYG{o}{\PYGZgt{}}
    \PYG{o}{|}            \PYG{o}{|}                 \PYG{o}{|}                  \PYG{o}{|}\PYG{o}{\PYGZhy{}}\PYG{o}{\PYGZhy{}}\PYG{o}{\PYGZhy{}}\PYG{o}{\PYGZhy{}}\PYG{o}{\PYGZhy{}}\PYG{o}{\PYGZhy{}}\PYG{o}{\PYGZhy{}}\PYG{o}{\PYGZhy{}}\PYG{o}{\PYGZhy{}}\PYG{o}{\PYGZhy{}} \PYG{p}{[}\PYG{o}{.}\PYG{o}{.}\PYG{o}{.}\PYG{p}{]}
    \PYG{o}{|}            \PYG{o}{|}                 \PYG{o}{|}\PYG{o}{\PYGZhy{}}\PYG{o}{\PYGZhy{}}\PYG{o}{\PYGZhy{}}\PYG{o}{\PYGZhy{}}\PYG{o}{\PYGZhy{}}\PYG{o}{\PYGZhy{}}\PYG{o}{\PYGZhy{}}\PYG{o}{\PYGZhy{}}\PYG{o}{\PYGZhy{}}\PYG{o}{\PYGZhy{}}\PYG{o}{\PYGZhy{}} \PYG{p}{[}\PYG{o}{.}\PYG{o}{.}\PYG{o}{.}\PYG{p}{]}
    \PYG{o}{|}            \PYG{o}{|}
    \PYG{o}{|}            \PYG{o}{|}\PYG{o}{\PYGZhy{}}\PYG{o}{\PYGZhy{}}\PYG{o}{\PYGZhy{}}\PYG{o}{\PYGZhy{}}\PYG{o}{\PYGZhy{}}\PYG{o}{\PYGZhy{}}\PYG{o}{\PYGZhy{}}\PYG{o}{\PYGZhy{}}\PYG{o}{\PYGZhy{}} \PYG{n}{Post}\PYG{o}{\PYGZhy{}}\PYG{n}{Processing}
    \PYG{o}{|}                             \PYG{o}{|}\PYG{o}{\PYGZhy{}}\PYG{o}{\PYGZhy{}}\PYG{o}{\PYGZhy{}}\PYG{o}{\PYGZhy{}}\PYG{o}{\PYGZhy{}}\PYG{o}{\PYGZhy{}}\PYG{o}{\PYGZhy{}}\PYG{o}{\PYGZhy{}}\PYG{o}{\PYGZhy{}}\PYG{o}{\PYGZhy{}}\PYG{o}{\PYGZhy{}}\PYG{o}{\PYGZhy{}} \PYG{n}{Comparisons}
    \PYG{o}{|}                             \PYG{o}{|}                  \PYG{o}{|}\PYG{o}{\PYGZhy{}}\PYG{o}{\PYGZhy{}}\PYG{o}{\PYGZhy{}}\PYG{o}{\PYGZhy{}}\PYG{o}{\PYGZhy{}}\PYG{o}{\PYGZhy{}}\PYG{o}{\PYGZhy{}}\PYG{o}{\PYGZhy{}} \PYG{o}{\PYGZlt{}}\PYG{n}{lib} \PYG{l+m+mi}{1}\PYG{o}{\PYGZgt{}}\PYG{n}{\PYGZus{}Vs\PYGZus{}}\PYG{o}{\PYGZlt{}}\PYG{n}{lib} \PYG{l+m+mi}{2}\PYG{o}{\PYGZgt{}}\PYG{n}{\PYGZus{}Vs}\PYG{o}{.}\PYG{o}{.}\PYG{o}{.}\PYG{o}{.}
    \PYG{o}{|}                             \PYG{o}{|}                  \PYG{o}{|}                  \PYG{o}{|}\PYG{o}{\PYGZhy{}}\PYG{o}{\PYGZhy{}}\PYG{o}{\PYGZhy{}}\PYG{o}{\PYGZhy{}}\PYG{o}{\PYGZhy{}}\PYG{o}{\PYGZhy{}}\PYG{o}{\PYGZhy{}}\PYG{o}{\PYGZhy{}}\PYG{o}{\PYGZhy{}}\PYG{o}{\PYGZhy{}}\PYG{o}{\PYGZhy{}}\PYG{o}{\PYGZhy{}}\PYG{o}{\PYGZhy{}}\PYG{o}{\PYGZhy{}}\PYG{o}{\PYGZhy{}}\PYG{o}{\PYGZhy{}}\PYG{o}{\PYGZhy{}}\PYG{o}{\PYGZhy{}} \PYG{o}{\PYGZlt{}}\PYG{n}{Benchmark} \PYG{n}{name} \PYG{l+m+mi}{1}\PYG{o}{\PYGZgt{}}
    \PYG{o}{|}                             \PYG{o}{|}                  \PYG{o}{|}                  \PYG{o}{|}                            \PYG{o}{|}\PYG{o}{\PYGZhy{}}\PYG{o}{\PYGZhy{}}\PYG{o}{\PYGZhy{}}\PYG{o}{\PYGZhy{}}\PYG{o}{\PYGZhy{}}\PYG{o}{\PYGZhy{}}\PYG{o}{\PYGZhy{}}\PYG{o}{\PYGZhy{}}\PYG{o}{\PYGZhy{}}\PYG{o}{\PYGZhy{}}\PYG{o}{\PYGZhy{}} \PYG{n}{Atlas}
    \PYG{o}{|}                             \PYG{o}{|}                  \PYG{o}{|}                  \PYG{o}{|}                            \PYG{o}{|}\PYG{o}{\PYGZhy{}}\PYG{o}{\PYGZhy{}}\PYG{o}{\PYGZhy{}}\PYG{o}{\PYGZhy{}}\PYG{o}{\PYGZhy{}}\PYG{o}{\PYGZhy{}}\PYG{o}{\PYGZhy{}}\PYG{o}{\PYGZhy{}}\PYG{o}{\PYGZhy{}}\PYG{o}{\PYGZhy{}}\PYG{o}{\PYGZhy{}} \PYG{n}{Excel}
    \PYG{o}{|}                             \PYG{o}{|}                  \PYG{o}{|}                  \PYG{o}{|}
    \PYG{o}{|}                             \PYG{o}{|}                  \PYG{o}{|}                  \PYG{o}{|}\PYG{o}{\PYGZhy{}}\PYG{o}{\PYGZhy{}}\PYG{o}{\PYGZhy{}}\PYG{o}{\PYGZhy{}}\PYG{o}{\PYGZhy{}}\PYG{o}{\PYGZhy{}}\PYG{o}{\PYGZhy{}}\PYG{o}{\PYGZhy{}}\PYG{o}{\PYGZhy{}}\PYG{o}{\PYGZhy{}}\PYG{o}{\PYGZhy{}}\PYG{o}{\PYGZhy{}}\PYG{o}{\PYGZhy{}}\PYG{o}{\PYGZhy{}}\PYG{o}{\PYGZhy{}}\PYG{o}{\PYGZhy{}}\PYG{o}{\PYGZhy{}}\PYG{o}{\PYGZhy{}} \PYG{p}{[}\PYG{o}{.}\PYG{o}{.}\PYG{o}{.}\PYG{p}{]}
    \PYG{o}{|}                             \PYG{o}{|}                  \PYG{o}{|}
    \PYG{o}{|}                             \PYG{o}{|}                  \PYG{o}{|}\PYG{o}{\PYGZhy{}}\PYG{o}{\PYGZhy{}}\PYG{o}{\PYGZhy{}}\PYG{o}{\PYGZhy{}}\PYG{o}{\PYGZhy{}}\PYG{o}{\PYGZhy{}}\PYG{o}{\PYGZhy{}} \PYG{p}{[}\PYG{o}{.}\PYG{o}{.}\PYG{o}{.}\PYG{p}{]}
    \PYG{o}{|}                             \PYG{o}{|}
    \PYG{o}{|}                             \PYG{o}{|}\PYG{o}{\PYGZhy{}}\PYG{o}{\PYGZhy{}}\PYG{o}{\PYGZhy{}}\PYG{o}{\PYGZhy{}}\PYG{o}{\PYGZhy{}}\PYG{o}{\PYGZhy{}}\PYG{o}{\PYGZhy{}}\PYG{o}{\PYGZhy{}}\PYG{o}{\PYGZhy{}}\PYG{o}{\PYGZhy{}}\PYG{o}{\PYGZhy{}}\PYG{o}{\PYGZhy{}} \PYG{n}{Single} \PYG{n}{Libraries}
    \PYG{o}{|}                                                 \PYG{o}{|}\PYG{o}{\PYGZhy{}}\PYG{o}{\PYGZhy{}}\PYG{o}{\PYGZhy{}}\PYG{o}{\PYGZhy{}}\PYG{o}{\PYGZhy{}}\PYG{o}{\PYGZhy{}}\PYG{o}{\PYGZhy{}}\PYG{o}{\PYGZhy{}}\PYG{o}{\PYGZhy{}}\PYG{o}{\PYGZhy{}}\PYG{o}{\PYGZhy{}} \PYG{o}{\PYGZlt{}}\PYG{n}{Lib} \PYG{n}{suffix} \PYG{l+m+mi}{1}\PYG{o}{\PYGZgt{}}
    \PYG{o}{|}                                                 \PYG{o}{|}                  \PYG{o}{|}\PYG{o}{\PYGZhy{}}\PYG{o}{\PYGZhy{}}\PYG{o}{\PYGZhy{}}\PYG{o}{\PYGZhy{}}\PYG{o}{\PYGZhy{}}\PYG{o}{\PYGZhy{}}\PYG{o}{\PYGZhy{}}\PYG{o}{\PYGZhy{}}\PYG{o}{\PYGZhy{}}\PYG{o}{\PYGZhy{}} \PYG{o}{\PYGZlt{}}\PYG{n}{Benchmark} \PYG{n}{name} \PYG{l+m+mi}{1}\PYG{o}{\PYGZgt{}}
    \PYG{o}{|}                                                 \PYG{o}{|}                  \PYG{o}{|}                   \PYG{o}{|}\PYG{o}{\PYGZhy{}}\PYG{o}{\PYGZhy{}}\PYG{o}{\PYGZhy{}}\PYG{o}{\PYGZhy{}}\PYG{o}{\PYGZhy{}}\PYG{o}{\PYGZhy{}}\PYG{o}{\PYGZhy{}}\PYG{o}{\PYGZhy{}}\PYG{o}{\PYGZhy{}}\PYG{o}{\PYGZhy{}}\PYG{o}{\PYGZhy{}} \PYG{n}{Atlas}
    \PYG{o}{|}                                                 \PYG{o}{|}                  \PYG{o}{|}                   \PYG{o}{|}\PYG{o}{\PYGZhy{}}\PYG{o}{\PYGZhy{}}\PYG{o}{\PYGZhy{}}\PYG{o}{\PYGZhy{}}\PYG{o}{\PYGZhy{}}\PYG{o}{\PYGZhy{}}\PYG{o}{\PYGZhy{}}\PYG{o}{\PYGZhy{}}\PYG{o}{\PYGZhy{}}\PYG{o}{\PYGZhy{}}\PYG{o}{\PYGZhy{}} \PYG{n}{Excel}
    \PYG{o}{|}                                                 \PYG{o}{|}                  \PYG{o}{|}                   \PYG{o}{|}\PYG{o}{\PYGZhy{}}\PYG{o}{\PYGZhy{}}\PYG{o}{\PYGZhy{}}\PYG{o}{\PYGZhy{}}\PYG{o}{\PYGZhy{}}\PYG{o}{\PYGZhy{}}\PYG{o}{\PYGZhy{}}\PYG{o}{\PYGZhy{}}\PYG{o}{\PYGZhy{}}\PYG{o}{\PYGZhy{}}\PYG{o}{\PYGZhy{}} \PYG{n}{Raw} \PYG{n}{Data}
    \PYG{o}{|}                                                 \PYG{o}{|}                  \PYG{o}{|}
    \PYG{o}{|}                                                 \PYG{o}{|}                  \PYG{o}{|}\PYG{o}{\PYGZhy{}}\PYG{o}{\PYGZhy{}}\PYG{o}{\PYGZhy{}}\PYG{o}{\PYGZhy{}}\PYG{o}{\PYGZhy{}}\PYG{o}{\PYGZhy{}}\PYG{o}{\PYGZhy{}}\PYG{o}{\PYGZhy{}}\PYG{o}{\PYGZhy{}}\PYG{o}{\PYGZhy{}} \PYG{p}{[}\PYG{o}{.}\PYG{o}{.}\PYG{o}{.}\PYG{p}{]}
    \PYG{o}{|}                                                 \PYG{o}{|}
    \PYG{o}{|}                                                 \PYG{o}{|}\PYG{o}{\PYGZhy{}}\PYG{o}{\PYGZhy{}}\PYG{o}{\PYGZhy{}}\PYG{o}{\PYGZhy{}}\PYG{o}{\PYGZhy{}}\PYG{o}{\PYGZhy{}}\PYG{o}{\PYGZhy{}}\PYG{o}{\PYGZhy{}}\PYG{o}{\PYGZhy{}}\PYG{o}{\PYGZhy{}}\PYG{o}{\PYGZhy{}} \PYG{p}{[}\PYG{o}{.}\PYG{o}{.}\PYG{o}{.}\PYG{p}{]}
    \PYG{o}{|}
    \PYG{o}{|}\PYG{o}{\PYGZhy{}}\PYG{o}{\PYGZhy{}}\PYG{o}{\PYGZhy{}}\PYG{o}{\PYGZhy{}}\PYG{o}{\PYGZhy{}}\PYG{o}{\PYGZhy{}}\PYG{o}{\PYGZhy{}}\PYG{o}{\PYGZhy{}} \PYG{n}{Utilities}
\end{sphinxVerbatim}

\sphinxcode{\sphinxupquote{\textless{}JADE\_root\textgreater{}}} is the root folder chosen by the user. As described in {\hyperref[\detokenize{usage/installation:install}]{\sphinxcrossref{\DUrole{std,std-ref}{Installation}}}} section,
the JADE GitHub repo should be renamed and placed inside the root directory as \sphinxcode{\sphinxupquote{\textless{}JADE\_root\textgreater{}\textbackslash{}Code}}.

All folders parallel to the \sphinxcode{\sphinxupquote{\textless{}JADE\_root\textgreater{}\textbackslash{}Code}} will be created after the first JADE run.

Hereafter, a general overview of the different JADE tree branches is presented.


\section{Benchmark inputs}
\label{\detokenize{usage/folders:benchmark-inputs}}
\sphinxcode{\sphinxupquote{\textless{}JADE\_root\textgreater{}\textbackslash{}Benchmark inputs}} contains all the inputs of the default benchmarks avaialble in the JADE suite.
This is the folder where eventual user defined benchmark inputs should be positioned.
In case of benchmarks that are composed by more than one run, all the inputs are reunited in a sub\sphinxhyphen{}folder
(e.g. \sphinxcode{\sphinxupquote{\textless{}JADE\_root\textgreater{}\textbackslash{}Benchmark inputs\textbackslash{}Oktavian}}.


\section{Code}
\label{\detokenize{usage/folders:code}}
\sphinxcode{\sphinxupquote{\textless{}JADE\_root\textgreater{}\textbackslash{}Code}} contains the JADE GitHub repo itself. All the source code is contained here toghether with the
following subfolders:
\begin{description}
\item[{Default Settings}] \leavevmode
Contains all JADE default settings. On the first JADE instance these are copied to the \sphinxcode{\sphinxupquote{\textless{}JADE\_root\textgreater{}\textbackslash{}Configuration}}
folder. They can be restored by a dedicated utility function available from the main menu.

\item[{docs}] \leavevmode
Contains all files related to this documentation. Here, local version of the documentations can be found.

\item[{Installation Files}] \leavevmode
Contains files to be used during the first JADE run. They have not any appeal to the general user.

\item[{Templates}] \leavevmode
Contains all the Microsoft Office and Word templates to be used during post\sphinxhyphen{}processing. In case of user\sphinxhyphen{}defined
benchmarks that are based on specific templates, these need to be added here.

\end{description}


\section{Configuration}
\label{\detokenize{usage/folders:configuration}}
\sphinxcode{\sphinxupquote{\textless{}JADE\_root\textgreater{}\textbackslash{}Configuration}} stores the main JADE configuration file \sphinxcode{\sphinxupquote{Config.xlsx}} and all benchmark\sphinxhyphen{}specific configuration
files that are stored in \sphinxcode{\sphinxupquote{\textless{}JADE\_root\textgreater{}\textbackslash{}Code\textbackslash{}Benchmarks Configuration}}.


\sphinxstrong{See also:}


{\hyperref[\detokenize{usage/configuration:config}]{\sphinxcrossref{\DUrole{std,std-ref}{Configuration}}}} for additional description of the configuration files.




\section{Experimental results}
\label{\detokenize{usage/folders:experimental-results}}
\sphinxcode{\sphinxupquote{\textless{}JADE\_root\textgreater{}\textbackslash{}Experimental results}} stores all the experimental results needed for the post\sphinxhyphen{}processing of
experimental benchmarks. In case of benchmarks that are composed by more than one run, all the inputs are reunited in a sub\sphinxhyphen{}folder
(e.g. \sphinxcode{\sphinxupquote{\textless{}JADE\_root\textgreater{}\textbackslash{}Experimental results\textbackslash{}Oktavian}}.


\section{Quality}
\label{\detokenize{usage/folders:quality}}
\sphinxstylestrong{NOT IMPLEMENTED}


\section{Tests}
\label{\detokenize{usage/folders:tests}}
\sphinxcode{\sphinxupquote{\textless{}JADE\_root\textgreater{}\textbackslash{}Tests}} reunites all the outputs of the benchmarks assessments.
\begin{description}
\item[{MCNP simulations}] \leavevmode
contains the results of the transport simulations.

\item[{Post\sphinxhyphen{}Processing}] \leavevmode
contains all the results coming from the post\sphinxhyphen{}processing of results. These are divided between
\sphinxstyleemphasis{Comparisons} and \sphinxstyleemphasis{Single Libraries}.

\end{description}


\section{Utilities}
\label{\detokenize{usage/folders:utilities}}
\sphinxcode{\sphinxupquote{\textless{}JADE\_root\textgreater{}\textbackslash{}Tests}} is where all outputs coming from the {\hyperref[\detokenize{usage/utilities:uty}]{\sphinxcrossref{\DUrole{std,std-ref}{Utilities}}}} are reunited. Each utility generates
a dedicated sub\sphinxhyphen{}folder when is used for the first time. Upon installation, the only sub\sphinxhyphen{}folder is
\sphinxcode{\sphinxupquote{\textless{}JADE\_root\textgreater{}\textbackslash{}Tests\textbackslash{}Log Files}} that contains all log files produced by each JADE session.


\chapter{Configuration}
\label{\detokenize{usage/configuration:configuration}}\label{\detokenize{usage/configuration:config}}\label{\detokenize{usage/configuration::doc}}
All configuration files are included in the \sphinxcode{\sphinxupquote{\textless{}JADE\_root\textgreater{}\textbackslash{}Configuration}} directory.
In principle, \sphinxstylestrong{the general user should only operate on the} {\hyperref[\detokenize{usage/configuration:mainconfig}]{\sphinxcrossref{\DUrole{std,std-ref}{Main Configuration}}}} \sphinxstylestrong{file}, while
all other configuration files simply guarantee an additional level of personalization for the user.

\begin{sphinxadmonition}{note}{Note:}
In case of user\sphinxhyphen{}defined benchmarks suitable {\hyperref[\detokenize{usage/configuration:runconf}]{\sphinxcrossref{\DUrole{std,std-ref}{Benchmark run configuration}}}} and {\hyperref[\detokenize{usage/configuration:ppconf}]{\sphinxcrossref{\DUrole{std,std-ref}{Benchmark post\sphinxhyphen{}processing configuration}}}} files need
to be produced.
\end{sphinxadmonition}


\section{Main Configuration}
\label{\detokenize{usage/configuration:main-configuration}}\label{\detokenize{usage/configuration:mainconfig}}
The most important configuration file is \sphinxcode{\sphinxupquote{\textless{}JADE\_root\textgreater{}\textbackslash{}Configuration\textbackslash{}Config.xlsx}}.
This is \sphinxstylestrong{the only configuration file that the user must modify} before operating with JADE.
Herafter, a description of the different sheets included in the file is given.


\subsection{MAIN Config.}
\label{\detokenize{usage/configuration:main-config}}
\noindent\sphinxincludegraphics[width=600\sphinxpxdimen]{{main}.png}

This sheet contains the JADE \sphinxstyleemphasis{ambient variables}:
\begin{description}
\item[{xsdir Path}] \leavevmode
Absolute path to the xsdir file that has been set to be used during MCNP simulations.

\item[{multithread}] \leavevmode
Under Windows operative system, MCNP allows to run on multithread using the \sphinxcode{\sphinxupquote{tasks}}
keyword. Setting this variable to \sphinxcode{\sphinxupquote{True}} enables this capability.

\item[{CPU}] \leavevmode
When \sphinxstylestrong{multithread} is set to \sphinxcode{\sphinxupquote{True}}, \sphinxstylestrong{CPU} sets the argument that will be passed
to \sphinxcode{\sphinxupquote{tasks}} during MCNP runs.

\end{description}


\subsection{Computational benchmarks}
\label{\detokenize{usage/configuration:computational-benchmarks}}\label{\detokenize{usage/configuration:compsheet}}
\noindent\sphinxincludegraphics[width=600\sphinxpxdimen]{{comp}.png}

This table collects allows to personalize which \sphinxstyleemphasis{computational benchmarks} should be included
in the JADE assessment. Each row controls a different benchmark, where the following options
(columns) are available:
\begin{description}
\item[{Description}] \leavevmode
this is the extended name of the benchmark, this name will appear in specific outputs of the
post\sphinxhyphen{}processing.

\item[{File Name}] \leavevmode
name of the reference MCNP input file. These need to be placed in \sphinxcode{\sphinxupquote{\textless{}JADE root\textgreater{}\textbackslash{}Benchmarks inputs}}.

\item[{OnlyInput}] \leavevmode
when this field is set to \sphinxcode{\sphinxupquote{True}} the benchmark input is only generated but not run. This can be
useful when the user wants to run the benchmark on a different hardware with respect to the
one where JADE is being used.


\sphinxstrong{See also:}


{\hyperref[\detokenize{usage/tipstricks:externalrun}]{\sphinxcrossref{\DUrole{std,std-ref}{External Run of a benchmark}}}}



\item[{Run}] \leavevmode
the benchmark will be run during an assessment only if this field is set to \sphinxcode{\sphinxupquote{True}}.
This allows to customize the selection of benchmarks to be run during an assessment or avoid
to re\sphinxhyphen{}run benchmarks that were already simulated in the past.

\item[{Post\sphinxhyphen{}Processing}] \leavevmode
this field works exactly as the \sphinxcode{\sphinxupquote{Run}} one but for the post\sphinxhyphen{}processing operations.

\end{description}

The last three options available for each benchmark control the MCNP STOP card parameters
that help regulating the simulation lenght:
\begin{description}
\item[{NPS cut\sphinxhyphen{}off}] \leavevmode
this is equivalent to the \sphinxcode{\sphinxupquote{NPS}} entry in the MCNP STOP card. It sets a maximum amount
of histories to be simulated. Only integers are allowed.

\item[{CTME cut\sphinxhyphen{}off}] \leavevmode
this is equivalent to the \sphinxcode{\sphinxupquote{CTME}} entry in the MCNP STOP card. It sets a maximum computer
time after which the simulation will be interrupted. Only integers are allowed.

\item[{Relative Error cut\sphinxhyphen{}off}] \leavevmode
this is equivalent to the \sphinxcode{\sphinxupquote{F}} entry in the MCNP STOP card. The sintax of this entry is:

F\textless{}\sphinxstyleemphasis{k}\textgreater{}\sphinxhyphen{}\textless{}\sphinxstyleemphasis{e}\textgreater{}  (example: F16\sphinxhyphen{}0.0005)

This stops the calculation when the tally fluctuation chart of tally \sphinxstyleemphasis{k} has reached a
relative error lower than \sphinxstyleemphasis{e}.

\end{description}

\begin{sphinxadmonition}{note}{Note:}
All three STOP parameters can be simultaneously defined during a simulation. The first
cut\sphinxhyphen{}off criteria reached will be the one triggering the end of the calculation.
\end{sphinxadmonition}


\subsection{Experimental benchmarks}
\label{\detokenize{usage/configuration:experimental-benchmarks}}
\noindent\sphinxincludegraphics[width=600\sphinxpxdimen]{{exp}.png}

The structure of the sheet is exactly the same as the {\hyperref[\detokenize{usage/configuration:compsheet}]{\sphinxcrossref{\DUrole{std,std-ref}{Computational benchmarks}}}} one. Nevertheless,
in this table are indicated the settings for the experimental benchmarks.


\subsection{Libraries}
\label{\detokenize{usage/configuration:libraries}}
\noindent\sphinxincludegraphics[width=400\sphinxpxdimen]{{lib}.png}

This table simply consists of a glossary where the user can associate more explicit
names to the nuclear data libraries suffixes available in the xsdir file. This
allows for a clearer post\sphinxhyphen{}processing output.


\section{Benchmark run configuration}
\label{\detokenize{usage/configuration:benchmark-run-configuration}}\label{\detokenize{usage/configuration:runconf}}
TBD

These are used only for \sphinxstyleemphasis{Sphere Leakage} and cannot be generalized.


\section{Benchmark post\sphinxhyphen{}processing configuration}
\label{\detokenize{usage/configuration:benchmark-post-processing-configuration}}\label{\detokenize{usage/configuration:ppconf}}
It is possible to control (to some extent) the post\sphinxhyphen{}processing of each benchmark via its
specific configuration file. These files are located in the \sphinxcode{\sphinxupquote{\textless{}JADE\_root\textgreater{}\textbackslash{}Configuration\textbackslash{}Benchmarks Configuration}}
folder and their name must be identical to the one used in the \sphinxcode{\sphinxupquote{File Name}} field in the main configuration file
(using the .xlsx extension instead of the .i). These files are available only for computational benchmarks,
since the high degree of customization needed for an experimental benchmark makes quite difficult to
standardize them. While computational benchmarks can be added to the JADE suite without the need for additional
coding, this is not true also for experimental one.

The files contain two main sheets, that respectively regulate the Excel and the Word/PDF post\sphinxhyphen{}processing output.


\subsection{Excel}
\label{\detokenize{usage/configuration:excel}}
\noindent\sphinxincludegraphics[width=600\sphinxpxdimen]{{excelbench}.png}

This sheet regulates the Excel output derived from the benchmark. It consists of a table where each row regulates
the output of a single tally present in the MCNP input.

Hereinafter a description of the available fields is reported.
\begin{description}
\item[{Tally}] \leavevmode
tally number according to MCNP input file.

\item[{x, y}] \leavevmode
select the binnings to be used for the presentation of the excel results of the specific tally. Clearly,
the binning should have been coherently defined in the MCNP input too. MCNP allows different types of tally binning,
they can be accessed using the tags reported in the table below.


\begin{savenotes}\sphinxattablestart
\centering
\sphinxcapstartof{table}
\sphinxthecaptionisattop
\sphinxcaption{Allowed binnings}\label{\detokenize{usage/configuration:id1}}
\sphinxaftertopcaption
\begin{tabular}[t]{|\X{50}{50}|}
\hline
\sphinxstyletheadfamily 
Admissible \sphinxstylestrong{x} and \sphinxstylestrong{y}
\\
\hline
Energy
\\
\hline
Cells
\\
\hline
time
\\
\hline
tally
\\
\hline
Dir
\\
\hline
User
\\
\hline
Segments
\\
\hline
Multiplier
\\
\hline
Cosine
\\
\hline
Cor A
\\
\hline
Cor B
\\
\hline
Cor C
\\
\hline
\end{tabular}
\par
\sphinxattableend\end{savenotes}

As a result of the selcted \sphinxstylestrong{x} and \sphinxstylestrong{y} option, the results of the post\sphinxhyphen{}processed tally will be display in a
matrix format. In case only a single binning is defined in the MCNP input, the \sphinxcode{\sphinxupquote{tally}} keyword should be used to
signal to JADE to just to print the results in a column format.

\begin{sphinxadmonition}{important}{Important:}
The main direction of an Excel file is considered to be the vertical one, which is the preferred scrolling direction.
For this reason, the \sphinxstylestrong{x} direction is associated with the vertical direction in an Excel file and the \sphinxstylestrong{y} with
the horizontal one.
\end{sphinxadmonition}

\begin{sphinxadmonition}{warning}{Warning:}
No more than two binnings should be defined for a single MCNP tally due to the limitation of having to represent
2\sphinxhyphen{}D output. JADE may be able to to handle tallies with more than 2 binnings if some of them are constant
values.
\end{sphinxadmonition}

\begin{sphinxadmonition}{tip}{Tip:}
If a 1D FMESH is defined in the MCNP input, JADE will automatically transform it to a “binned” tally and handle it
as any other tally using the \sphinxcode{\sphinxupquote{Cor A}}, \sphinxcode{\sphinxupquote{Cor B}} or \sphinxcode{\sphinxupquote{Cor C}} field.
\end{sphinxadmonition}

\item[{x name, y name}] \leavevmode
These will be the names associated to the \sphinxstylestrong{x} and \sphinxstylestrong{y} axis printed in the excel file.

\item[{cut Y}] \leavevmode
The idea behind JADE is to produce outputs that are easy to investigate simply by scrolling and concentrate on the
main results highlighted through colors. Having a high number of bins both in the x and y axis may cause a problem
in this sense, forcing the user to scroll on both axis. For this reason, a maximum number of columns can be set to
solve this issue. This will cause the tally results not to be printed as a unique matrix but as sequential blocks
each with a number of columns equal to \sphinxstylestrong{cut Y}.

\end{description}


\subsection{Atlas}
\label{\detokenize{usage/configuration:atlas}}
\noindent\sphinxincludegraphics[width=600\sphinxpxdimen]{{atlasbench}.png}

This sheet regulates the Atlas output (Word/PDF) derived from the benchmark. It consists of a table where each row regulates
the output of a single tally present in the MCNP input.
Hereinafter a description of the available fields is reported.
\begin{description}
\item[{Tally}] \leavevmode
tally number according to MCNP input file.

\item[{Quantity}] \leavevmode
Physical quantity that will be plotted on the y\sphinxhyphen{}axis of the plot. For the x\sphinxhyphen{}axis the one specified in the Excel sheet
under \sphinxstylestrong{x} will be considered. The quantity selected for plotting will always be the tallied quantity.

\begin{sphinxadmonition}{important}{Important:}
when two binnings are specified in the Excel sheet, a different plot for each of the \sphinxstylestrong{y} bins will be produced.
For example, let’s consider a neutron flux tally binned both in energy (selected as \sphinxstylestrong{x}) and cells (selected as \sphinxstylestrong{y}).
Then, a plot showing the neutron flux as a function of energy will be produced for each cell indicated in the tally.
\end{sphinxadmonition}

\item[{Unit}] \leavevmode
Unit associated to the Quantity.

\item[{\textless{}Graph type\textgreater{}}] \leavevmode
Different columns can be added where it can be specified if a plot in the style indicated by the column name
should be generated (\sphinxcode{\sphinxupquote{true}}) or not (\sphinxcode{\sphinxupquote{false}}). The available plot styles are \sphinxstyleemphasis{Binned graph}, \sphinxstyleemphasis{Ratio Graph},
\sphinxstyleemphasis{Experimental points} and \sphinxstyleemphasis{Grouped bars}.


\sphinxstrong{See also:}


{\hyperref[\detokenize{usage/postprocessing:plotstyles}]{\sphinxcrossref{\DUrole{std,std-ref}{Plots Atlas}}}} for an additional description of the available plot styles.



\end{description}


\chapter{Usage}
\label{\detokenize{usage/menu:usage}}\label{\detokenize{usage/menu:menu}}\label{\detokenize{usage/menu::doc}}
Once JADE is correctly configured
(for additional details see {\hyperref[\detokenize{usage/configuration:config}]{\sphinxcrossref{\DUrole{std,std-ref}{Configuration}}}}), the application can be started
from the ‘Code’ folder with:

\sphinxcode{\sphinxupquote{python main.py}}

The main menu is loaded at this point:

\noindent\sphinxincludegraphics[width=400\sphinxpxdimen]{{main_menu}.png}

This menu allows users to interact with the tool directly from the
command prompt via pre\sphinxhyphen{}defined commands.
The following main option are available typing from the main menu:
\begin{itemize}
\item {} 
\sphinxcode{\sphinxupquote{qual}} not currently supported;

\item {} 
\sphinxcode{\sphinxupquote{comp}} opens the {\hyperref[\detokenize{usage/menu:compmenu}]{\sphinxcrossref{\DUrole{std,std-ref}{Computational Benchmark menu}}}};

\item {} 
\sphinxcode{\sphinxupquote{exp}} opens the {\hyperref[\detokenize{usage/menu:expmenu}]{\sphinxcrossref{\DUrole{std,std-ref}{Experimental Benchmark menu}}}};

\item {} 
\sphinxcode{\sphinxupquote{post}} opens the {\hyperref[\detokenize{usage/menu:postmenu}]{\sphinxcrossref{\DUrole{std,std-ref}{Post\sphinxhyphen{}processing menu}}}};

\item {} 
\sphinxcode{\sphinxupquote{exit}} exit the application.

\end{itemize}

Additionaly to these main options, a series of “utilities” can be also accessed
from the main menu. These are a collection of side\sphinxhyphen{}tools initially developed
for JADE that nevertheless can be useful also as stand\sphinxhyphen{}alone tools.
A detailed description of these functionalities can be found in {\hyperref[\detokenize{usage/utilities:uty}]{\sphinxcrossref{\DUrole{std,std-ref}{Utilities}}}}.


\section{Quality check menu}
\label{\detokenize{usage/menu:quality-check-menu}}
Not implemented.


\section{Computational Benchmark menu}
\label{\detokenize{usage/menu:computational-benchmark-menu}}\label{\detokenize{usage/menu:compmenu}}
\noindent\sphinxincludegraphics[width=400\sphinxpxdimen]{{compmenu}.png}

The following options are available in the computational benchmark menu:
\begin{itemize}
\item {} 
\sphinxcode{\sphinxupquote{printlib}} print to video all the available nuclear data libraries
in the xsdir file selected during JADE configuration;

\item {} 
\sphinxcode{\sphinxupquote{assess}} start the assessment of a selected library on the computational benchmarks. The library is
specified directly from the console when the selection is prompted to
video. The library must be contained in the xsdir file (available libraries
can be explored using \sphinxcode{\sphinxupquote{printlib}});

\item {} 
\sphinxcode{\sphinxupquote{continue}} continue a previously interrupted assessment for a selected
library. \sphinxstylestrong{Currently, this option is implemented only for the Sphere Leakage
benchmark.}

\item {} 
\sphinxcode{\sphinxupquote{back}} go back to the main menu;

\item {} 
\sphinxcode{\sphinxupquote{exit}} exit the application.

\end{itemize}

The selection of the libraries is done indicating their correspondent suffix specified in the xsdir file
(e.g. \sphinxcode{\sphinxupquote{31c}}).

\begin{sphinxadmonition}{note}{Note:}
Whenever an assessment is requested, all the benchmarks selected in the main configuration file will be considered.
In case the requested library was already assesed on one or more of the active benchmarks,
the user will be asked for permission before overriding the results.
\end{sphinxadmonition}


\sphinxstrong{See also:}


{\hyperref[\detokenize{usage/configuration:config}]{\sphinxcrossref{\DUrole{std,std-ref}{Configuration}}}} for additional details on the benchmark selection.




\section{Experimental Benchmark menu}
\label{\detokenize{usage/menu:experimental-benchmark-menu}}\label{\detokenize{usage/menu:expmenu}}
\noindent\sphinxincludegraphics[width=400\sphinxpxdimen]{{expmenu}.png}

The following options are available in the experimental benchmark menu:
\begin{itemize}
\item {} 
\sphinxcode{\sphinxupquote{printlib}} print to video all the available nuclear data libraries
in the xsdir file selected during JADE configuration;

\item {} 
\sphinxcode{\sphinxupquote{assess}} start the assessment of a selected library on the experimental benchmarks. The library is
specified directly from the console when the selection is prompted to
video. The library must be contained in the xsdir file (available libraries
can be explored using \sphinxcode{\sphinxupquote{printlib}});

\item {} 
\sphinxcode{\sphinxupquote{continue}} \sphinxstylestrong{{[}not implemented{]}}

\item {} 
\sphinxcode{\sphinxupquote{back}} go back to the main menu;

\item {} 
\sphinxcode{\sphinxupquote{exit}} exit the application.

\end{itemize}

The selection of the libraries is done indicating their correspondent suffix specified in the xsdir file
(e.g. \sphinxcode{\sphinxupquote{31c}}).

\begin{sphinxadmonition}{note}{Note:}
Whenever an assessment is requested, all the benchmarks selected in the main configuration file will be considered.
In case the requested library was already assesed on one or more of the active benchmarks,
the user will be asked for permission before overriding the results.
\end{sphinxadmonition}


\sphinxstrong{See also:}


{\hyperref[\detokenize{usage/configuration:config}]{\sphinxcrossref{\DUrole{std,std-ref}{Configuration}}}} for additional details on the benchmark selection.




\section{Post\sphinxhyphen{}processing menu}
\label{\detokenize{usage/menu:post-processing-menu}}\label{\detokenize{usage/menu:postmenu}}
\noindent\sphinxincludegraphics[width=400\sphinxpxdimen]{{postmenu}.png}

The following options are available in the post\sphinxhyphen{}processing menu:
\begin{itemize}
\item {} 
\sphinxcode{\sphinxupquote{printlib}} print all libraries that were tested and that are available for post\sphinxhyphen{}processing;

\item {} 
\sphinxcode{\sphinxupquote{pp}} post\sphinxhyphen{}process a single library;

\item {} 
\sphinxcode{\sphinxupquote{compare}} compare different libraries results on computational benchmarks;

\item {} 
\sphinxcode{\sphinxupquote{compexp}} compare different libraries results on experimental benchmarks;

\item {} 
\sphinxcode{\sphinxupquote{back}} go back to the main menu;

\item {} 
\sphinxcode{\sphinxupquote{exit}} exit the application.

\end{itemize}

For the \sphinxcode{\sphinxupquote{pp}}, \sphinxcode{\sphinxupquote{compare}} and \sphinxcode{\sphinxupquote{compexp}} the selection of the libraries will be directly prompt to video.
The selection of the libraries is done indicating their correspondent suffix specified in the xsdir file
(e.g. \sphinxcode{\sphinxupquote{31c}}). When comparing more than one library, the suffixes should be separated by a ‘\sphinxhyphen{}‘ (e.g. \sphinxcode{\sphinxupquote{31c\sphinxhyphen{}32c}}).
The first library that is indicated is always considered as the \sphinxstyleemphasis{reference library} for the post\sphinxhyphen{}processing.
There may be a limitation on the number of libraries that can be compared at once depending on the post\sphinxhyphen{}processing settings.

Only one library at the time can be post\sphinxhyphen{}processed with the \sphinxcode{\sphinxupquote{pp}} option. Nevertheless, when a comparison is requested that
includes libraries that were not singularly post\sphinxhyphen{}processed, an automatic \sphinxcode{\sphinxupquote{pp}} operation is conducted on them.

\begin{sphinxadmonition}{warning}{Warning:}
Please note that \sphinxcode{\sphinxupquote{printlib}} will simply show all libraries for which at least one benchmark has been run.
\end{sphinxadmonition}

\begin{sphinxadmonition}{warning}{Warning:}
Please note that part of the single post\sphinxhyphen{}processing of the libraries is used in the comparisons. Also, JADE does not perform
any checks on the consistency between the two. This responsability is left to the user.
The following is an example of incorrect usage that can lead to erroneous results:
\begin{enumerate}
\sphinxsetlistlabels{\arabic}{enumi}{enumii}{}{.}%
\item {} 
a first assessment is run;

\item {} 
single post\sphinxhyphen{}processing is completed;

\item {} 
some configuration settings are changed and the assessment is re\sphinxhyphen{}run;

\item {} 
a comparison is requested.

\end{enumerate}

In this case, JADE cannot know that the first single post\sphinxhyphen{}processing was done on a different benchmark run wiith respect
to the requested comparison. As a result, the outputs coming from different assessments will be mixed up.
\end{sphinxadmonition}

\begin{sphinxadmonition}{note}{Note:}
Whenever a post\sphinxhyphen{}processing is requested, all the benchmarks selected in the main configuration file will be considered.
In case one or more of the requested libraries were already post\sphinxhyphen{}processed on one or more of the active benchmarks,
the user will be asked for permission before overriding the post\sphinxhyphen{}processing results.
\end{sphinxadmonition}


\sphinxstrong{See also:}


{\hyperref[\detokenize{usage/configuration:config}]{\sphinxcrossref{\DUrole{std,std-ref}{Configuration}}}} for additional details on the benchmark selection.




\chapter{Default Benchmarks}
\label{\detokenize{usage/benchmarks:default-benchmarks}}\label{\detokenize{usage/benchmarks::doc}}

\section{Computational Benchmarks}
\label{\detokenize{usage/benchmarks:computational-benchmarks}}

\subsection{Sphere Leakage}
\label{\detokenize{usage/benchmarks:sphere-leakage}}
\begin{figure}[htbp]
\centering
\capstart

\noindent\sphinxincludegraphics{{sphere}.png}
\caption{Sphere Leakage geometrical model}\label{\detokenize{usage/benchmarks:id1}}\end{figure}

The Sphere Leakage benchmark is arguably the most important
benchmark included in the JADE suite. Indeed, it allows to test
individually each single isotope of the nuclear data library under assessment
plus some typically used material in the ITER project namely:
\begin{itemize}
\item {} 
Water;

\item {} 
Ordinary Concrete;

\item {} 
Boron Carbide;

\item {} 
SS316L(N)\sphinxhyphen{}IG;

\item {} 
Natural Silicon;

\item {} 
Polyethylene (non\sphinxhyphen{}borated).

\end{itemize}

The Sphere Lekage geometry consists of actually three
concentric spheres. The inner one is void and has a radius of 5 cm. Here
is located the uniform probability 0\sphinxhyphen{}14 MeV neutron point source. The second sphere
has a radius of 50 cm and it is composed entirely by a single isotope
or a typical ITER material. Finally,
the last 60 cm radius sphere acts as a graveyard where particles importance is
set to zero and the boundary of the model is defined.

\sphinxstylestrong{TBD: Describe tallies}


\sphinxstrong{See also:}


\sphinxstylestrong{Related papers and contributions:}
\begin{itemize}
\item {} 
D. Laghi, M. Fabbri, L. Isolan, R. Pampin, M. Sumini, A. Portone and
A. Trkov, 2020,
“JADE, a new software tool for nuclear fusion data libraries verification \&
validation”, \sphinxstyleemphasis{Fusion Engineering and Design}, \sphinxstylestrong{161} 112075

\end{itemize}




\subsection{ITER 1D}
\label{\detokenize{usage/benchmarks:iter-1d}}
\begin{figure}[htbp]
\centering
\capstart

\noindent\sphinxincludegraphics[width=600\sphinxpxdimen]{{iter1D}.png}
\caption{ITER 1D MCNP geometry (quarter)}\label{\detokenize{usage/benchmarks:id2}}\end{figure}


\sphinxstrong{See also:}


\sphinxstylestrong{Related papers and contributions:}
\begin{itemize}
\item {} 
M. Sawan, 1994,  “FENDL Neutronics Benchmark: Specifications for the calculational and shielding benchmark”,
(Vienna: INDC(NDS)\sphinxhyphen{}316)

\end{itemize}




\subsection{Test Blanket Module}
\label{\detokenize{usage/benchmarks:test-blanket-module}}
\begin{figure}[htbp]
\centering
\capstart

\noindent\sphinxincludegraphics[width=600\sphinxpxdimen]{{ITERCAD}.png}
\caption{Position of the MCNP model in the ITER tokamak}\label{\detokenize{usage/benchmarks:id3}}\end{figure}

\begin{figure}[htbp]
\centering
\capstart

\noindent\sphinxincludegraphics[width=600\sphinxpxdimen]{{tbmCAD}.png}
\caption{CAD model of the TBM component}\label{\detokenize{usage/benchmarks:id4}}\end{figure}


\subsubsection{HCPB TBM in ITER 1D}
\label{\detokenize{usage/benchmarks:hcpb-tbm-in-iter-1d}}
\begin{figure}[htbp]
\centering
\capstart

\noindent\sphinxincludegraphics[width=600\sphinxpxdimen]{{HCPBcad}.png}
\caption{Section of the CAD model of the HCPB TBM set}\label{\detokenize{usage/benchmarks:id5}}\end{figure}

\begin{figure}[htbp]
\centering
\capstart

\noindent\sphinxincludegraphics[width=600\sphinxpxdimen]{{HCPBmcnp}.png}
\caption{Visualization of the TBM and and shielding section in the 1D MCNP geometry}\label{\detokenize{usage/benchmarks:id6}}\end{figure}


\subsubsection{WCLL TBM in ITER 1D}
\label{\detokenize{usage/benchmarks:wcll-tbm-in-iter-1d}}
\begin{figure}[htbp]
\centering
\capstart

\noindent\sphinxincludegraphics[width=600\sphinxpxdimen]{{WCLLcad}.png}
\caption{Section of the CAD model of the WCLL TBM set}\label{\detokenize{usage/benchmarks:id7}}\end{figure}

\begin{figure}[htbp]
\centering
\capstart

\noindent\sphinxincludegraphics[width=600\sphinxpxdimen]{{WCLLmcnp}.png}
\caption{Visualization of the TBM and shielding section in the 1D MCNP geometry}\label{\detokenize{usage/benchmarks:id8}}\end{figure}


\subsubsection{C\sphinxhyphen{}Model}
\label{\detokenize{usage/benchmarks:c-model}}
This benchmark input cannot be distributed directly with JADE. The user must request to obtain it
from ITER organization and insert it in the \sphinxcode{\sphinxupquote{\textless{}JADE root\textgreater{}\textbackslash{}Benchmarks inputs}} directory renaming it
‘C\_Model.i’.

\begin{sphinxadmonition}{important}{Important:}
The NPS card needs to be removed from the input. It is recommended to also delete total bins
from standard tallies for a clearer post\sphinxhyphen{}processing results.
\end{sphinxadmonition}

\begin{figure}[htbp]
\centering
\capstart

\noindent\sphinxincludegraphics[width=600\sphinxpxdimen]{{cmodel}.png}
\caption{C\sphinxhyphen{}model R181031. Origin (1050,200,0). Basis (0.982339, 0.187112, 0.000000)
(0,0,1). Extent (1000,1000)}\label{\detokenize{usage/benchmarks:id9}}\end{figure}


\sphinxstrong{See also:}


\sphinxstylestrong{Related papers and contributions:}
\begin{itemize}
\item {} 
D. Leichtle, B. Colling, M. Fabbri, R. Juarez, M. Loughlin,
R. Pampin, E. Polunovskiy, A. Serikov, A. Turner and L. Bertalot, 2018,
“The ITER tokamak neutronics reference model C\sphinxhyphen{}Model”,
\sphinxstyleemphasis{Fusion Engineering and Design}, \sphinxstylestrong{136} 742\sphinxhyphen{}746

\end{itemize}




\section{Experimental Benchmarks}
\label{\detokenize{usage/benchmarks:experimental-benchmarks}}

\subsection{Oktavian}
\label{\detokenize{usage/benchmarks:oktavian}}
\begin{figure}[htbp]
\centering
\capstart

\noindent\sphinxincludegraphics[width=600\sphinxpxdimen]{{oktaviansimplified}.png}
\caption{Simplified layout of the OKTAVIAN Fe experimental setup (not on scale).}\label{\detokenize{usage/benchmarks:id10}}\end{figure}

Experimental results derived from Oktavian experiments are publicly accessible at the
\sphinxhref{https://www-nds.iaea.org/conderc/oktavian}{CoNDERC database} which is mantained by the
IAEA Nuclear Data Section and built upon the
\sphinxhref{https://rsicc.ornl.gov/Benchmarks.aspx}{database of shielding experiments} (SINBAD), hosted
by the RSICC and jointly mantained with the NEA data bank.


\sphinxstrong{See also:}


\sphinxstylestrong{Related papers and contributions:}
\begin{itemize}
\item {} 
A. Milocco, A. Trkov and I. A. Kodeli, 2010, “The OKTAVIAN TOF experiments in SINBAD: Evaluation of the
experimental uncertainties”, \sphinxstyleemphasis{Annals of Nuclear Energy}, \sphinxstylestrong{37} 443\sphinxhyphen{}449

\item {} 
I.Kodeli, E. Sartori and B. Kirk, “SINBAD \sphinxhyphen{} Shielding Benchmark Experiments \sphinxhyphen{} Status and Planned Activities”,
\sphinxstyleemphasis{Proceedings of the ANS 14th Biennial Topical Meeting of Radiation Protection and Shielding Division},
Carlsbad, New Mexico (April 3\sphinxhyphen{}6, 2006)

\end{itemize}




\chapter{Post\sphinxhyphen{}Processing Gallery}
\label{\detokenize{usage/postprocessing:post-processing-gallery}}\label{\detokenize{usage/postprocessing::doc}}

\section{Excel output}
\label{\detokenize{usage/postprocessing:excel-output}}

\subsection{Benchmark specific}
\label{\detokenize{usage/postprocessing:benchmark-specific}}

\subsubsection{Sphere Leakage}
\label{\detokenize{usage/postprocessing:sphere-leakage}}
\begin{figure}[htbp]
\centering
\capstart

\noindent\sphinxincludegraphics[width=600\sphinxpxdimen]{{statchecks_sphere}.png}
\caption{10 MCNP statistical checks for each zaid and tally}\label{\detokenize{usage/postprocessing:id1}}\end{figure}

\begin{figure}[htbp]
\centering
\capstart

\noindent\sphinxincludegraphics[width=600\sphinxpxdimen]{{errors_sphere}.png}
\caption{Statistical error associated with each tally for each zaid}\label{\detokenize{usage/postprocessing:id2}}\end{figure}

\begin{figure}[htbp]
\centering
\capstart

\noindent\sphinxincludegraphics[width=600\sphinxpxdimen]{{consistencysphere}.png}
\caption{Consistency checks on zaid tally results}\label{\detokenize{usage/postprocessing:id3}}\end{figure}

\begin{figure}[htbp]
\centering
\capstart

\noindent\sphinxincludegraphics[width=600\sphinxpxdimen]{{comparisonsphere}.png}
\caption{Comparison of tally results for each zaid}\label{\detokenize{usage/postprocessing:id4}}\end{figure}


\subsubsection{Oktavian}
\label{\detokenize{usage/postprocessing:oktavian}}
\begin{figure}[htbp]
\centering
\capstart

\noindent\sphinxincludegraphics[width=600\sphinxpxdimen]{{oktavexcel}.png}
\caption{C/E table summarized per energy range}\label{\detokenize{usage/postprocessing:id5}}\end{figure}


\subsection{General output}
\label{\detokenize{usage/postprocessing:general-output}}
\begin{figure}[htbp]
\centering
\capstart

\noindent\sphinxincludegraphics[width=600\sphinxpxdimen]{{statchecksgeneral}.png}
\caption{10 MCNP statistical checks recap}\label{\detokenize{usage/postprocessing:id6}}\end{figure}

\begin{figure}[htbp]
\centering
\capstart

\noindent\sphinxincludegraphics[width=600\sphinxpxdimen]{{errors}.png}
\caption{Statistical errors associated with the tally results}\label{\detokenize{usage/postprocessing:id7}}\end{figure}

\begin{figure}[htbp]
\centering
\capstart

\noindent\sphinxincludegraphics[width=600\sphinxpxdimen]{{column}.png}
\caption{Print of a single binned tally}\label{\detokenize{usage/postprocessing:id8}}\end{figure}

\begin{figure}[htbp]
\centering
\capstart

\noindent\sphinxincludegraphics[width=600\sphinxpxdimen]{{matrix}.png}
\caption{Print of a double binned tally}\label{\detokenize{usage/postprocessing:id9}}\end{figure}


\section{Plots Atlas}
\label{\detokenize{usage/postprocessing:plots-atlas}}\label{\detokenize{usage/postprocessing:plotstyles}}

\subsection{Binned graph}
\label{\detokenize{usage/postprocessing:binned-graph}}
\noindent\sphinxincludegraphics[width=600\sphinxpxdimen]{{binned}.png}


\subsection{Ratio Graph}
\label{\detokenize{usage/postprocessing:ratio-graph}}
\noindent\sphinxincludegraphics[width=600\sphinxpxdimen]{{ratio}.png}

\noindent\sphinxincludegraphics[width=600\sphinxpxdimen]{{ratio2}.png}


\subsection{Experimental points}
\label{\detokenize{usage/postprocessing:experimental-points}}
\noindent\sphinxincludegraphics[width=600\sphinxpxdimen]{{exp1}.png}


\subsection{Grouped bars}
\label{\detokenize{usage/postprocessing:grouped-bars}}
\noindent\sphinxincludegraphics[width=600\sphinxpxdimen]{{grouped}.png}


\chapter{Utilities}
\label{\detokenize{usage/utilities:utilities}}\label{\detokenize{usage/utilities:uty}}\label{\detokenize{usage/utilities::doc}}
During the development of JADE, many useful classes and methods were developed
which could be used for small stand\sphinxhyphen{}alone tools, mostly
operating on MCNP inputs.

A description of these \sphinxstyleemphasis{utilities}, accessible from the JADE main menu,
is here provided.

The outputs (if generated) of these utilities can be found in specific subfolders
of the \sphinxcode{\sphinxupquote{\textless{}JADE root\textgreater{}\textbackslash{}Utilities}} directory.


\section{Print available libraries}
\label{\detokenize{usage/utilities:print-available-libraries}}
\sphinxcode{\sphinxupquote{printlib}}

This function allows to print to video all libraries (suffixes) that are
available in xsdir file indicated in the main configuration file.

\begin{figure}[htbp]
\centering
\capstart

\noindent\sphinxincludegraphics[width=600\sphinxpxdimen]{{printlib}.png}
\caption{Screenshot of the execution of the \sphinxcode{\sphinxupquote{printlib}} command}\label{\detokenize{usage/utilities:id1}}\end{figure}


\section{Restore default configurations}
\label{\detokenize{usage/utilities:restore-default-configurations}}
\sphinxcode{\sphinxupquote{restore}}

This function allows to restore the JADE configuration default settings.
In other words, the content of the \sphinxcode{\sphinxupquote{\textless{}JADE root\textgreater{}\textbackslash{}Configuration}} directory
is restored to “factory installation” and all user modifications to the
configuration files are lost.

\begin{sphinxadmonition}{note}{Note:}
When the the restoration is completed, the application will be terminated.
The main configuration file ambient variable will need to be reconfigured
before running another JADE instance.
\end{sphinxadmonition}


\sphinxstrong{See also:}


{\hyperref[\detokenize{usage/configuration:mainconfig}]{\sphinxcrossref{\DUrole{std,std-ref}{Main Configuration}}}}




\section{Translate an MCNP input}
\label{\detokenize{usage/utilities:translate-an-mcnp-input}}
\sphinxcode{\sphinxupquote{trans}}

This function allows to translate a material section of an MCNP input to
a whatever nuclear data library available in the xsdir file.

The translation is carried out basically by the \sphinxstylestrong{convertZaid()} method of the
\sphinxstylestrong{LibManager} class and by the \sphinxstylestrong{translate()} method of the \sphinxstylestrong{SubMaterial} class.
The \sphinxstylestrong{convertZaid} method:
\begin{enumerate}
\sphinxsetlistlabels{\arabic}{enumi}{enumii}{}{.}%
\item {} 
asks for a zaid (to translate) and for a library (to translate to);

\item {} 
checks if the library selected for the translation is available in the xsdir of
the user;

\item {} \begin{description}
\item[{select the type of translation:}] \leavevmode\begin{enumerate}
\sphinxsetlistlabels{\alph}{enumii}{enumiii}{}{.}%
\item {} 
zaid not available in library: the default lib is used, no other changes
applied;

\item {} 
zaid available in library: the zaid is converted to the selected library,
no other changes applied;

\item {} 
the zaid is natural (i.e. it ends with 000).

\end{enumerate}

\end{description}

\end{enumerate}

For case c, at first, the selected library is checked for exact correspondence,
i.e., it is checked if also in the selected library the zaid is expressed as natural.
In this case, the behavior is identical to case b. If this is not true, the zaid needs
to be expanded: all zaids of the same elements are returned with their atomic mass (m)
and natural abundance (NA).

At this point, the \sphinxstylestrong{translate()} method completes the translation. No particular actions
are required if there is no zaid expansion.
In case of expansion, if the original natural zaid fraction is an atomic one
(\(x^A_N\)), the new zaids deriving from the expansion will have as fraction their
natural abundance (NA) multiplied for the original natural zaid fraction:
\begin{equation*}
\begin{split}x^A_{zaid} = \text{NA}_{zaid}\cdot x^A_N\end{split}
\end{equation*}
If, instead, the original natural zaid fraction is a mass one (\(x^M_N\)),
the \sphinxstyleemphasis{equivalent mass} \(m_N\) of the natural zaid can be computed as:
\begin{equation*}
\begin{split}m_N = \sum_{zaids} \text{NA}_{zaid}\cdot m_{zaid}\end{split}
\end{equation*}
and then the mass fraction of each expanded new zaid (\(x^M_\text{zaid}\))
can be calculated as:
\begin{equation*}
\begin{split}x^M_\text{zaid}=x^M_N\cdot (\text{NA}_{zaid}\cdot m_{zaid})/M_N\end{split}
\end{equation*}
where \((\text{NA}_{zaid}\cdot m_{zaid})/M_N\)
is basically the natural abundance in mass of the zaid.

The new input will be dumped in the
\sphinxcode{\sphinxupquote{\textless{}JADE root\textgreater{}\textbackslash{}Utilities\textbackslash{}Translation}} folder.
The following scheme summarizes the JADE translation logic.

\begin{figure}[htbp]
\centering
\capstart

\noindent\sphinxincludegraphics[width=600\sphinxpxdimen]{{Translation_logic}.jpg}
\caption{Zaid translation logic}\label{\detokenize{usage/utilities:id2}}\end{figure}


\section{Print materials info}
\label{\detokenize{usage/utilities:print-materials-info}}
\sphinxcode{\sphinxupquote{printmat}}

This function is used to print a summary of an MCNP input material section.
The information is contained in two sheets of an Excel file dumped into the
\sphinxcode{\sphinxupquote{\textless{}JADE root\textgreater{}\textbackslash{}Utilities\textbackslash{}Materials Infos}} folder.
The first sheet summarizes information at the single isotope level.
Here both the atom and mass fraction for each zaid is reported divided by
material and submaterial. It may happen that the original fraction appearing
in the MCNP input is not normalized. JADE prints this fraction as it is and
only the alternative fraction is normalized during its calculation.

\begin{figure}[htbp]
\centering
\capstart

\noindent\sphinxincludegraphics[width=600\sphinxpxdimen]{{printmat1}.png}
\caption{Extract of the isotope sheet. In the example, the material card was
expressed in mass fraction and not normalized.}\label{\detokenize{usage/utilities:id3}}\end{figure}

The second sheet summarizes information at the element level.
Three fractions are here listed for each element:
* the MCNP fraction of the element in the material;
* the normalized fraction of the element in the submaterial;
* the normalized fraction of the element in the material.

Depending on the orginal MCNP input, these three fraction need to be
interpreted as either \sphinxstyleemphasis{mass} or \sphinxstyleemphasis{atom} fraction.

\begin{figure}[htbp]
\centering
\capstart

\noindent\sphinxincludegraphics[width=600\sphinxpxdimen]{{printmat2}.png}
\caption{Extract of the element sheet. In the example, the material card was
expressed in mass fraction and not normalized.}\label{\detokenize{usage/utilities:id4}}\end{figure}


\section{Generate material mixture}
\label{\detokenize{usage/utilities:generate-material-mixture}}
\sphinxcode{\sphinxupquote{generate}}

This function is used to generate a material mixture starting from two or
more materials contained in a single MCNP input. The user will be asked for:
\begin{itemize}
\item {} 
absolute path to the MCNP input;

\item {} 
if the zaids need to have a mass or atom fraction;

\item {} 
material names (e.g. m1) to be used in the mixture;

\item {} 
percentages to be used in the mixture for each material;

\item {} 
nuclear data library to use for the new material mixture.

\end{itemize}

Each material will be transformed in a submaterial of the newly generated mixture
retaining its header if present. The new material will be dumped in the
\sphinxcode{\sphinxupquote{\textless{}JADE root\textgreater{}\textbackslash{}Utilities\textbackslash{}Generated Materials}} folder.


\section{Switch material fractions}
\label{\detokenize{usage/utilities:switch-material-fractions}}
\sphinxcode{\sphinxupquote{switch}}

This function can be used to switch an MCNP input from having atom fractions
to mass fractions and viceversa. The new input will be dumped in the
\sphinxcode{\sphinxupquote{\textless{}JADE root\textgreater{}\textbackslash{}Utilities\textbackslash{}Fraction switch}} folder.


\chapter{Tips \& Tricks}
\label{\detokenize{usage/tipstricks:tips-tricks}}\label{\detokenize{usage/tipstricks::doc}}
This section reunites a series of tips and tricks that can be used to \sphinxstyleemphasis{unlock}
JADE additional capabilities.


\section{External Run of a benchmark}
\label{\detokenize{usage/tipstricks:external-run-of-a-benchmark}}\label{\detokenize{usage/tipstricks:externalrun}}
It may be useful for particularly computational\sphinxhyphen{}intensive benchmark to be
run on a separate hardware (e.g. a server) with respect to the one used for JADE.
This can be achieved quite easily with the following steps:
\begin{enumerate}
\sphinxsetlistlabels{\arabic}{enumi}{enumii}{}{.}%
\item {} 
set the \sphinxcode{\sphinxupquote{OnlyInput}} option in the \sphinxcode{\sphinxupquote{\textless{}JADE root\textgreater{}\textbackslash{}Configuration\textbackslash{}Conf.xlsx}}
file to \sphinxcode{\sphinxupquote{True}} for the benchmark that needs to be run externally. This
will generate the MCNP input file of the benchmark that can be found in
\sphinxcode{\sphinxupquote{\textless{}JADE root\textgreater{}\textbackslash{}Tests\textbackslash{}MCNP simulation\textbackslash{}\textless{}lib suffix\textgreater{}\textbackslash{}\textless{}Benchmark name\textgreater{}}}
without running it;

\item {} 
copy the generated input file into the hardware selected for the run and start the
MCNP simulation. The only requirement is to use the MCNP keyword  \sphinxcode{\sphinxupquote{name=}}
when launching the simulation in order to obtain consistently named outputs;

\item {} 
once the simulation is completed, copy all MCNP outputs to the same
\sphinxcode{\sphinxupquote{\textless{}JADE root\textgreater{}\textbackslash{}Tests\textbackslash{}MCNP simulation\textbackslash{}\textless{}lib suffix\textgreater{}\textbackslash{}\textless{}Benchmark name\textgreater{}}} folder;

\item {} 
normally run the post\sphinxhyphen{}processing.

\end{enumerate}


\chapter{JADE Testing}
\label{\detokenize{testing/testing:jade-testing}}\label{\detokenize{testing/testing::doc}}
TBD


\chapter{License}
\label{\detokenize{LICENSE:license}}\label{\detokenize{LICENSE::doc}}
JADE software is licensed under the {\hyperref[\detokenize{LICENSE:gnulicense}]{\sphinxcrossref{\DUrole{std,std-ref}{GNU GPLv3 License}}}}.

The following external python modules are re\sphinxhyphen{}distributed toghether with
JADE and their licenses needs to be propagated:
\begin{itemize}
\item {} 
\sphinxstylestrong{MCTAL\_READER.py}, licensed under the {\hyperref[\detokenize{LICENSE:gnulicense}]{\sphinxcrossref{\DUrole{std,std-ref}{GNU GPLv3 License}}}};

\item {} 
\sphinxstylestrong{xsdirpyne.py}, licensed under the {\hyperref[\detokenize{LICENSE:pynelicense}]{\sphinxcrossref{\DUrole{std,std-ref}{Pyne License}}}}.

\end{itemize}


\section{GNU GPLv3 License}
\label{\detokenize{LICENSE:gnu-gplv3-license}}\label{\detokenize{LICENSE:gnulicense}}
\begin{sphinxVerbatim}[commandchars=\\\{\}]
                    GNU GENERAL PUBLIC LICENSE
                       Version 3, 29 June 2007

 Copyright (C) 2007 Free Software Foundation, Inc. \PYGZlt{}https://fsf.org/\PYGZgt{}
 Everyone is permitted to copy and distribute verbatim copies
 of this license document, but changing it is not allowed.

                            Preamble

  The GNU General Public License is a free, copyleft license for
software and other kinds of works.

  The licenses for most software and other practical works are designed
to take away your freedom to share and change the works.  By contrast,
the GNU General Public License is intended to guarantee your freedom to
share and change all versions of a program\PYGZhy{}\PYGZhy{}to make sure it remains free
software for all its users.  We, the Free Software Foundation, use the
GNU General Public License for most of our software; it applies also to
any other work released this way by its authors.  You can apply it to
your programs, too.

  When we speak of free software, we are referring to freedom, not
price.  Our General Public Licenses are designed to make sure that you
have the freedom to distribute copies of free software (and charge for
them if you wish), that you receive source code or can get it if you
want it, that you can change the software or use pieces of it in new
free programs, and that you know you can do these things.

  To protect your rights, we need to prevent others from denying you
these rights or asking you to surrender the rights.  Therefore, you have
certain responsibilities if you distribute copies of the software, or if
you modify it: responsibilities to respect the freedom of others.

  For example, if you distribute copies of such a program, whether
gratis or for a fee, you must pass on to the recipients the same
freedoms that you received.  You must make sure that they, too, receive
or can get the source code.  And you must show them these terms so they
know their rights.

  Developers that use the GNU GPL protect your rights with two steps:
(1) assert copyright on the software, and (2) offer you this License
giving you legal permission to copy, distribute and/or modify it.

  For the developers\PYGZsq{} and authors\PYGZsq{} protection, the GPL clearly explains
that there is no warranty for this free software.  For both users\PYGZsq{} and
authors\PYGZsq{} sake, the GPL requires that modified versions be marked as
changed, so that their problems will not be attributed erroneously to
authors of previous versions.

  Some devices are designed to deny users access to install or run
modified versions of the software inside them, although the manufacturer
can do so.  This is fundamentally incompatible with the aim of
protecting users\PYGZsq{} freedom to change the software.  The systematic
pattern of such abuse occurs in the area of products for individuals to
use, which is precisely where it is most unacceptable.  Therefore, we
have designed this version of the GPL to prohibit the practice for those
products.  If such problems arise substantially in other domains, we
stand ready to extend this provision to those domains in future versions
of the GPL, as needed to protect the freedom of users.

  Finally, every program is threatened constantly by software patents.
States should not allow patents to restrict development and use of
software on general\PYGZhy{}purpose computers, but in those that do, we wish to
avoid the special danger that patents applied to a free program could
make it effectively proprietary.  To prevent this, the GPL assures that
patents cannot be used to render the program non\PYGZhy{}free.

  The precise terms and conditions for copying, distribution and
modification follow.

                       TERMS AND CONDITIONS

  0. Definitions.

  \PYGZdq{}This License\PYGZdq{} refers to version 3 of the GNU General Public License.

  \PYGZdq{}Copyright\PYGZdq{} also means copyright\PYGZhy{}like laws that apply to other kinds of
works, such as semiconductor masks.

  \PYGZdq{}The Program\PYGZdq{} refers to any copyrightable work licensed under this
License.  Each licensee is addressed as \PYGZdq{}you\PYGZdq{}.  \PYGZdq{}Licensees\PYGZdq{} and
\PYGZdq{}recipients\PYGZdq{} may be individuals or organizations.

  To \PYGZdq{}modify\PYGZdq{} a work means to copy from or adapt all or part of the work
in a fashion requiring copyright permission, other than the making of an
exact copy.  The resulting work is called a \PYGZdq{}modified version\PYGZdq{} of the
earlier work or a work \PYGZdq{}based on\PYGZdq{} the earlier work.

  A \PYGZdq{}covered work\PYGZdq{} means either the unmodified Program or a work based
on the Program.

  To \PYGZdq{}propagate\PYGZdq{} a work means to do anything with it that, without
permission, would make you directly or secondarily liable for
infringement under applicable copyright law, except executing it on a
computer or modifying a private copy.  Propagation includes copying,
distribution (with or without modification), making available to the
public, and in some countries other activities as well.

  To \PYGZdq{}convey\PYGZdq{} a work means any kind of propagation that enables other
parties to make or receive copies.  Mere interaction with a user through
a computer network, with no transfer of a copy, is not conveying.

  An interactive user interface displays \PYGZdq{}Appropriate Legal Notices\PYGZdq{}
to the extent that it includes a convenient and prominently visible
feature that (1) displays an appropriate copyright notice, and (2)
tells the user that there is no warranty for the work (except to the
extent that warranties are provided), that licensees may convey the
work under this License, and how to view a copy of this License.  If
the interface presents a list of user commands or options, such as a
menu, a prominent item in the list meets this criterion.

  1. Source Code.

  The \PYGZdq{}source code\PYGZdq{} for a work means the preferred form of the work
for making modifications to it.  \PYGZdq{}Object code\PYGZdq{} means any non\PYGZhy{}source
form of a work.

  A \PYGZdq{}Standard Interface\PYGZdq{} means an interface that either is an official
standard defined by a recognized standards body, or, in the case of
interfaces specified for a particular programming language, one that
is widely used among developers working in that language.

  The \PYGZdq{}System Libraries\PYGZdq{} of an executable work include anything, other
than the work as a whole, that (a) is included in the normal form of
packaging a Major Component, but which is not part of that Major
Component, and (b) serves only to enable use of the work with that
Major Component, or to implement a Standard Interface for which an
implementation is available to the public in source code form.  A
\PYGZdq{}Major Component\PYGZdq{}, in this context, means a major essential component
(kernel, window system, and so on) of the specific operating system
(if any) on which the executable work runs, or a compiler used to
produce the work, or an object code interpreter used to run it.

  The \PYGZdq{}Corresponding Source\PYGZdq{} for a work in object code form means all
the source code needed to generate, install, and (for an executable
work) run the object code and to modify the work, including scripts to
control those activities.  However, it does not include the work\PYGZsq{}s
System Libraries, or general\PYGZhy{}purpose tools or generally available free
programs which are used unmodified in performing those activities but
which are not part of the work.  For example, Corresponding Source
includes interface definition files associated with source files for
the work, and the source code for shared libraries and dynamically
linked subprograms that the work is specifically designed to require,
such as by intimate data communication or control flow between those
subprograms and other parts of the work.

  The Corresponding Source need not include anything that users
can regenerate automatically from other parts of the Corresponding
Source.

  The Corresponding Source for a work in source code form is that
same work.

  2. Basic Permissions.

  All rights granted under this License are granted for the term of
copyright on the Program, and are irrevocable provided the stated
conditions are met.  This License explicitly affirms your unlimited
permission to run the unmodified Program.  The output from running a
covered work is covered by this License only if the output, given its
content, constitutes a covered work.  This License acknowledges your
rights of fair use or other equivalent, as provided by copyright law.

  You may make, run and propagate covered works that you do not
convey, without conditions so long as your license otherwise remains
in force.  You may convey covered works to others for the sole purpose
of having them make modifications exclusively for you, or provide you
with facilities for running those works, provided that you comply with
the terms of this License in conveying all material for which you do
not control copyright.  Those thus making or running the covered works
for you must do so exclusively on your behalf, under your direction
and control, on terms that prohibit them from making any copies of
your copyrighted material outside their relationship with you.

  Conveying under any other circumstances is permitted solely under
the conditions stated below.  Sublicensing is not allowed; section 10
makes it unnecessary.

  3. Protecting Users\PYGZsq{} Legal Rights From Anti\PYGZhy{}Circumvention Law.

  No covered work shall be deemed part of an effective technological
measure under any applicable law fulfilling obligations under article
11 of the WIPO copyright treaty adopted on 20 December 1996, or
similar laws prohibiting or restricting circumvention of such
measures.

  When you convey a covered work, you waive any legal power to forbid
circumvention of technological measures to the extent such circumvention
is effected by exercising rights under this License with respect to
the covered work, and you disclaim any intention to limit operation or
modification of the work as a means of enforcing, against the work\PYGZsq{}s
users, your or third parties\PYGZsq{} legal rights to forbid circumvention of
technological measures.

  4. Conveying Verbatim Copies.

  You may convey verbatim copies of the Program\PYGZsq{}s source code as you
receive it, in any medium, provided that you conspicuously and
appropriately publish on each copy an appropriate copyright notice;
keep intact all notices stating that this License and any
non\PYGZhy{}permissive terms added in accord with section 7 apply to the code;
keep intact all notices of the absence of any warranty; and give all
recipients a copy of this License along with the Program.

  You may charge any price or no price for each copy that you convey,
and you may offer support or warranty protection for a fee.

  5. Conveying Modified Source Versions.

  You may convey a work based on the Program, or the modifications to
produce it from the Program, in the form of source code under the
terms of section 4, provided that you also meet all of these conditions:

    a) The work must carry prominent notices stating that you modified
    it, and giving a relevant date.

    b) The work must carry prominent notices stating that it is
    released under this License and any conditions added under section
    7.  This requirement modifies the requirement in section 4 to
    \PYGZdq{}keep intact all notices\PYGZdq{}.

    c) You must license the entire work, as a whole, under this
    License to anyone who comes into possession of a copy.  This
    License will therefore apply, along with any applicable section 7
    additional terms, to the whole of the work, and all its parts,
    regardless of how they are packaged.  This License gives no
    permission to license the work in any other way, but it does not
    invalidate such permission if you have separately received it.

    d) If the work has interactive user interfaces, each must display
    Appropriate Legal Notices; however, if the Program has interactive
    interfaces that do not display Appropriate Legal Notices, your
    work need not make them do so.

  A compilation of a covered work with other separate and independent
works, which are not by their nature extensions of the covered work,
and which are not combined with it such as to form a larger program,
in or on a volume of a storage or distribution medium, is called an
\PYGZdq{}aggregate\PYGZdq{} if the compilation and its resulting copyright are not
used to limit the access or legal rights of the compilation\PYGZsq{}s users
beyond what the individual works permit.  Inclusion of a covered work
in an aggregate does not cause this License to apply to the other
parts of the aggregate.

  6. Conveying Non\PYGZhy{}Source Forms.

  You may convey a covered work in object code form under the terms
of sections 4 and 5, provided that you also convey the
machine\PYGZhy{}readable Corresponding Source under the terms of this License,
in one of these ways:

    a) Convey the object code in, or embodied in, a physical product
    (including a physical distribution medium), accompanied by the
    Corresponding Source fixed on a durable physical medium
    customarily used for software interchange.

    b) Convey the object code in, or embodied in, a physical product
    (including a physical distribution medium), accompanied by a
    written offer, valid for at least three years and valid for as
    long as you offer spare parts or customer support for that product
    model, to give anyone who possesses the object code either (1) a
    copy of the Corresponding Source for all the software in the
    product that is covered by this License, on a durable physical
    medium customarily used for software interchange, for a price no
    more than your reasonable cost of physically performing this
    conveying of source, or (2) access to copy the
    Corresponding Source from a network server at no charge.

    c) Convey individual copies of the object code with a copy of the
    written offer to provide the Corresponding Source.  This
    alternative is allowed only occasionally and noncommercially, and
    only if you received the object code with such an offer, in accord
    with subsection 6b.

    d) Convey the object code by offering access from a designated
    place (gratis or for a charge), and offer equivalent access to the
    Corresponding Source in the same way through the same place at no
    further charge.  You need not require recipients to copy the
    Corresponding Source along with the object code.  If the place to
    copy the object code is a network server, the Corresponding Source
    may be on a different server (operated by you or a third party)
    that supports equivalent copying facilities, provided you maintain
    clear directions next to the object code saying where to find the
    Corresponding Source.  Regardless of what server hosts the
    Corresponding Source, you remain obligated to ensure that it is
    available for as long as needed to satisfy these requirements.

    e) Convey the object code using peer\PYGZhy{}to\PYGZhy{}peer transmission, provided
    you inform other peers where the object code and Corresponding
    Source of the work are being offered to the general public at no
    charge under subsection 6d.

  A separable portion of the object code, whose source code is excluded
from the Corresponding Source as a System Library, need not be
included in conveying the object code work.

  A \PYGZdq{}User Product\PYGZdq{} is either (1) a \PYGZdq{}consumer product\PYGZdq{}, which means any
tangible personal property which is normally used for personal, family,
or household purposes, or (2) anything designed or sold for incorporation
into a dwelling.  In determining whether a product is a consumer product,
doubtful cases shall be resolved in favor of coverage.  For a particular
product received by a particular user, \PYGZdq{}normally used\PYGZdq{} refers to a
typical or common use of that class of product, regardless of the status
of the particular user or of the way in which the particular user
actually uses, or expects or is expected to use, the product.  A product
is a consumer product regardless of whether the product has substantial
commercial, industrial or non\PYGZhy{}consumer uses, unless such uses represent
the only significant mode of use of the product.

  \PYGZdq{}Installation Information\PYGZdq{} for a User Product means any methods,
procedures, authorization keys, or other information required to install
and execute modified versions of a covered work in that User Product from
a modified version of its Corresponding Source.  The information must
suffice to ensure that the continued functioning of the modified object
code is in no case prevented or interfered with solely because
modification has been made.

  If you convey an object code work under this section in, or with, or
specifically for use in, a User Product, and the conveying occurs as
part of a transaction in which the right of possession and use of the
User Product is transferred to the recipient in perpetuity or for a
fixed term (regardless of how the transaction is characterized), the
Corresponding Source conveyed under this section must be accompanied
by the Installation Information.  But this requirement does not apply
if neither you nor any third party retains the ability to install
modified object code on the User Product (for example, the work has
been installed in ROM).

  The requirement to provide Installation Information does not include a
requirement to continue to provide support service, warranty, or updates
for a work that has been modified or installed by the recipient, or for
the User Product in which it has been modified or installed.  Access to a
network may be denied when the modification itself materially and
adversely affects the operation of the network or violates the rules and
protocols for communication across the network.

  Corresponding Source conveyed, and Installation Information provided,
in accord with this section must be in a format that is publicly
documented (and with an implementation available to the public in
source code form), and must require no special password or key for
unpacking, reading or copying.

  7. Additional Terms.

  \PYGZdq{}Additional permissions\PYGZdq{} are terms that supplement the terms of this
License by making exceptions from one or more of its conditions.
Additional permissions that are applicable to the entire Program shall
be treated as though they were included in this License, to the extent
that they are valid under applicable law.  If additional permissions
apply only to part of the Program, that part may be used separately
under those permissions, but the entire Program remains governed by
this License without regard to the additional permissions.

  When you convey a copy of a covered work, you may at your option
remove any additional permissions from that copy, or from any part of
it.  (Additional permissions may be written to require their own
removal in certain cases when you modify the work.)  You may place
additional permissions on material, added by you to a covered work,
for which you have or can give appropriate copyright permission.

  Notwithstanding any other provision of this License, for material you
add to a covered work, you may (if authorized by the copyright holders of
that material) supplement the terms of this License with terms:

    a) Disclaiming warranty or limiting liability differently from the
    terms of sections 15 and 16 of this License; or

    b) Requiring preservation of specified reasonable legal notices or
    author attributions in that material or in the Appropriate Legal
    Notices displayed by works containing it; or

    c) Prohibiting misrepresentation of the origin of that material, or
    requiring that modified versions of such material be marked in
    reasonable ways as different from the original version; or

    d) Limiting the use for publicity purposes of names of licensors or
    authors of the material; or

    e) Declining to grant rights under trademark law for use of some
    trade names, trademarks, or service marks; or

    f) Requiring indemnification of licensors and authors of that
    material by anyone who conveys the material (or modified versions of
    it) with contractual assumptions of liability to the recipient, for
    any liability that these contractual assumptions directly impose on
    those licensors and authors.

  All other non\PYGZhy{}permissive additional terms are considered \PYGZdq{}further
restrictions\PYGZdq{} within the meaning of section 10.  If the Program as you
received it, or any part of it, contains a notice stating that it is
governed by this License along with a term that is a further
restriction, you may remove that term.  If a license document contains
a further restriction but permits relicensing or conveying under this
License, you may add to a covered work material governed by the terms
of that license document, provided that the further restriction does
not survive such relicensing or conveying.

  If you add terms to a covered work in accord with this section, you
must place, in the relevant source files, a statement of the
additional terms that apply to those files, or a notice indicating
where to find the applicable terms.

  Additional terms, permissive or non\PYGZhy{}permissive, may be stated in the
form of a separately written license, or stated as exceptions;
the above requirements apply either way.

  8. Termination.

  You may not propagate or modify a covered work except as expressly
provided under this License.  Any attempt otherwise to propagate or
modify it is void, and will automatically terminate your rights under
this License (including any patent licenses granted under the third
paragraph of section 11).

  However, if you cease all violation of this License, then your
license from a particular copyright holder is reinstated (a)
provisionally, unless and until the copyright holder explicitly and
finally terminates your license, and (b) permanently, if the copyright
holder fails to notify you of the violation by some reasonable means
prior to 60 days after the cessation.

  Moreover, your license from a particular copyright holder is
reinstated permanently if the copyright holder notifies you of the
violation by some reasonable means, this is the first time you have
received notice of violation of this License (for any work) from that
copyright holder, and you cure the violation prior to 30 days after
your receipt of the notice.

  Termination of your rights under this section does not terminate the
licenses of parties who have received copies or rights from you under
this License.  If your rights have been terminated and not permanently
reinstated, you do not qualify to receive new licenses for the same
material under section 10.

  9. Acceptance Not Required for Having Copies.

  You are not required to accept this License in order to receive or
run a copy of the Program.  Ancillary propagation of a covered work
occurring solely as a consequence of using peer\PYGZhy{}to\PYGZhy{}peer transmission
to receive a copy likewise does not require acceptance.  However,
nothing other than this License grants you permission to propagate or
modify any covered work.  These actions infringe copyright if you do
not accept this License.  Therefore, by modifying or propagating a
covered work, you indicate your acceptance of this License to do so.

  10. Automatic Licensing of Downstream Recipients.

  Each time you convey a covered work, the recipient automatically
receives a license from the original licensors, to run, modify and
propagate that work, subject to this License.  You are not responsible
for enforcing compliance by third parties with this License.

  An \PYGZdq{}entity transaction\PYGZdq{} is a transaction transferring control of an
organization, or substantially all assets of one, or subdividing an
organization, or merging organizations.  If propagation of a covered
work results from an entity transaction, each party to that
transaction who receives a copy of the work also receives whatever
licenses to the work the party\PYGZsq{}s predecessor in interest had or could
give under the previous paragraph, plus a right to possession of the
Corresponding Source of the work from the predecessor in interest, if
the predecessor has it or can get it with reasonable efforts.

  You may not impose any further restrictions on the exercise of the
rights granted or affirmed under this License.  For example, you may
not impose a license fee, royalty, or other charge for exercise of
rights granted under this License, and you may not initiate litigation
(including a cross\PYGZhy{}claim or counterclaim in a lawsuit) alleging that
any patent claim is infringed by making, using, selling, offering for
sale, or importing the Program or any portion of it.

  11. Patents.

  A \PYGZdq{}contributor\PYGZdq{} is a copyright holder who authorizes use under this
License of the Program or a work on which the Program is based.  The
work thus licensed is called the contributor\PYGZsq{}s \PYGZdq{}contributor version\PYGZdq{}.

  A contributor\PYGZsq{}s \PYGZdq{}essential patent claims\PYGZdq{} are all patent claims
owned or controlled by the contributor, whether already acquired or
hereafter acquired, that would be infringed by some manner, permitted
by this License, of making, using, or selling its contributor version,
but do not include claims that would be infringed only as a
consequence of further modification of the contributor version.  For
purposes of this definition, \PYGZdq{}control\PYGZdq{} includes the right to grant
patent sublicenses in a manner consistent with the requirements of
this License.

  Each contributor grants you a non\PYGZhy{}exclusive, worldwide, royalty\PYGZhy{}free
patent license under the contributor\PYGZsq{}s essential patent claims, to
make, use, sell, offer for sale, import and otherwise run, modify and
propagate the contents of its contributor version.

  In the following three paragraphs, a \PYGZdq{}patent license\PYGZdq{} is any express
agreement or commitment, however denominated, not to enforce a patent
(such as an express permission to practice a patent or covenant not to
sue for patent infringement).  To \PYGZdq{}grant\PYGZdq{} such a patent license to a
party means to make such an agreement or commitment not to enforce a
patent against the party.

  If you convey a covered work, knowingly relying on a patent license,
and the Corresponding Source of the work is not available for anyone
to copy, free of charge and under the terms of this License, through a
publicly available network server or other readily accessible means,
then you must either (1) cause the Corresponding Source to be so
available, or (2) arrange to deprive yourself of the benefit of the
patent license for this particular work, or (3) arrange, in a manner
consistent with the requirements of this License, to extend the patent
license to downstream recipients.  \PYGZdq{}Knowingly relying\PYGZdq{} means you have
actual knowledge that, but for the patent license, your conveying the
covered work in a country, or your recipient\PYGZsq{}s use of the covered work
in a country, would infringe one or more identifiable patents in that
country that you have reason to believe are valid.

  If, pursuant to or in connection with a single transaction or
arrangement, you convey, or propagate by procuring conveyance of, a
covered work, and grant a patent license to some of the parties
receiving the covered work authorizing them to use, propagate, modify
or convey a specific copy of the covered work, then the patent license
you grant is automatically extended to all recipients of the covered
work and works based on it.

  A patent license is \PYGZdq{}discriminatory\PYGZdq{} if it does not include within
the scope of its coverage, prohibits the exercise of, or is
conditioned on the non\PYGZhy{}exercise of one or more of the rights that are
specifically granted under this License.  You may not convey a covered
work if you are a party to an arrangement with a third party that is
in the business of distributing software, under which you make payment
to the third party based on the extent of your activity of conveying
the work, and under which the third party grants, to any of the
parties who would receive the covered work from you, a discriminatory
patent license (a) in connection with copies of the covered work
conveyed by you (or copies made from those copies), or (b) primarily
for and in connection with specific products or compilations that
contain the covered work, unless you entered into that arrangement,
or that patent license was granted, prior to 28 March 2007.

  Nothing in this License shall be construed as excluding or limiting
any implied license or other defenses to infringement that may
otherwise be available to you under applicable patent law.

  12. No Surrender of Others\PYGZsq{} Freedom.

  If conditions are imposed on you (whether by court order, agreement or
otherwise) that contradict the conditions of this License, they do not
excuse you from the conditions of this License.  If you cannot convey a
covered work so as to satisfy simultaneously your obligations under this
License and any other pertinent obligations, then as a consequence you may
not convey it at all.  For example, if you agree to terms that obligate you
to collect a royalty for further conveying from those to whom you convey
the Program, the only way you could satisfy both those terms and this
License would be to refrain entirely from conveying the Program.

  13. Use with the GNU Affero General Public License.

  Notwithstanding any other provision of this License, you have
permission to link or combine any covered work with a work licensed
under version 3 of the GNU Affero General Public License into a single
combined work, and to convey the resulting work.  The terms of this
License will continue to apply to the part which is the covered work,
but the special requirements of the GNU Affero General Public License,
section 13, concerning interaction through a network will apply to the
combination as such.

  14. Revised Versions of this License.

  The Free Software Foundation may publish revised and/or new versions of
the GNU General Public License from time to time.  Such new versions will
be similar in spirit to the present version, but may differ in detail to
address new problems or concerns.

  Each version is given a distinguishing version number.  If the
Program specifies that a certain numbered version of the GNU General
Public License \PYGZdq{}or any later version\PYGZdq{} applies to it, you have the
option of following the terms and conditions either of that numbered
version or of any later version published by the Free Software
Foundation.  If the Program does not specify a version number of the
GNU General Public License, you may choose any version ever published
by the Free Software Foundation.

  If the Program specifies that a proxy can decide which future
versions of the GNU General Public License can be used, that proxy\PYGZsq{}s
public statement of acceptance of a version permanently authorizes you
to choose that version for the Program.

  Later license versions may give you additional or different
permissions.  However, no additional obligations are imposed on any
author or copyright holder as a result of your choosing to follow a
later version.

  15. Disclaimer of Warranty.

  THERE IS NO WARRANTY FOR THE PROGRAM, TO THE EXTENT PERMITTED BY
APPLICABLE LAW.  EXCEPT WHEN OTHERWISE STATED IN WRITING THE COPYRIGHT
HOLDERS AND/OR OTHER PARTIES PROVIDE THE PROGRAM \PYGZdq{}AS IS\PYGZdq{} WITHOUT WARRANTY
OF ANY KIND, EITHER EXPRESSED OR IMPLIED, INCLUDING, BUT NOT LIMITED TO,
THE IMPLIED WARRANTIES OF MERCHANTABILITY AND FITNESS FOR A PARTICULAR
PURPOSE.  THE ENTIRE RISK AS TO THE QUALITY AND PERFORMANCE OF THE PROGRAM
IS WITH YOU.  SHOULD THE PROGRAM PROVE DEFECTIVE, YOU ASSUME THE COST OF
ALL NECESSARY SERVICING, REPAIR OR CORRECTION.

  16. Limitation of Liability.

  IN NO EVENT UNLESS REQUIRED BY APPLICABLE LAW OR AGREED TO IN WRITING
WILL ANY COPYRIGHT HOLDER, OR ANY OTHER PARTY WHO MODIFIES AND/OR CONVEYS
THE PROGRAM AS PERMITTED ABOVE, BE LIABLE TO YOU FOR DAMAGES, INCLUDING ANY
GENERAL, SPECIAL, INCIDENTAL OR CONSEQUENTIAL DAMAGES ARISING OUT OF THE
USE OR INABILITY TO USE THE PROGRAM (INCLUDING BUT NOT LIMITED TO LOSS OF
DATA OR DATA BEING RENDERED INACCURATE OR LOSSES SUSTAINED BY YOU OR THIRD
PARTIES OR A FAILURE OF THE PROGRAM TO OPERATE WITH ANY OTHER PROGRAMS),
EVEN IF SUCH HOLDER OR OTHER PARTY HAS BEEN ADVISED OF THE POSSIBILITY OF
SUCH DAMAGES.

  17. Interpretation of Sections 15 and 16.

  If the disclaimer of warranty and limitation of liability provided
above cannot be given local legal effect according to their terms,
reviewing courts shall apply local law that most closely approximates
an absolute waiver of all civil liability in connection with the
Program, unless a warranty or assumption of liability accompanies a
copy of the Program in return for a fee.

                     END OF TERMS AND CONDITIONS

            How to Apply These Terms to Your New Programs

  If you develop a new program, and you want it to be of the greatest
possible use to the public, the best way to achieve this is to make it
free software which everyone can redistribute and change under these terms.

  To do so, attach the following notices to the program.  It is safest
to attach them to the start of each source file to most effectively
state the exclusion of warranty; and each file should have at least
the \PYGZdq{}copyright\PYGZdq{} line and a pointer to where the full notice is found.

    \PYGZlt{}one line to give the program\PYGZsq{}s name and a brief idea of what it does.\PYGZgt{}
    Copyright (C) \PYGZlt{}year\PYGZgt{}  \PYGZlt{}name of author\PYGZgt{}

    This program is free software: you can redistribute it and/or modify
    it under the terms of the GNU General Public License as published by
    the Free Software Foundation, either version 3 of the License, or
    (at your option) any later version.

    This program is distributed in the hope that it will be useful,
    but WITHOUT ANY WARRANTY; without even the implied warranty of
    MERCHANTABILITY or FITNESS FOR A PARTICULAR PURPOSE.  See the
    GNU General Public License for more details.

    You should have received a copy of the GNU General Public License
    along with this program.  If not, see \PYGZlt{}https://www.gnu.org/licenses/\PYGZgt{}.

Also add information on how to contact you by electronic and paper mail.

  If the program does terminal interaction, make it output a short
notice like this when it starts in an interactive mode:

    \PYGZlt{}program\PYGZgt{}  Copyright (C) \PYGZlt{}year\PYGZgt{}  \PYGZlt{}name of author\PYGZgt{}
    This program comes with ABSOLUTELY NO WARRANTY; for details type `show w\PYGZsq{}.
    This is free software, and you are welcome to redistribute it
    under certain conditions; type `show c\PYGZsq{} for details.

The hypothetical commands `show w\PYGZsq{} and `show c\PYGZsq{} should show the appropriate
parts of the General Public License.  Of course, your program\PYGZsq{}s commands
might be different; for a GUI interface, you would use an \PYGZdq{}about box\PYGZdq{}.

  You should also get your employer (if you work as a programmer) or school,
if any, to sign a \PYGZdq{}copyright disclaimer\PYGZdq{} for the program, if necessary.
For more information on this, and how to apply and follow the GNU GPL, see
\PYGZlt{}https://www.gnu.org/licenses/\PYGZgt{}.

  The GNU General Public License does not permit incorporating your program
into proprietary programs.  If your program is a subroutine library, you
may consider it more useful to permit linking proprietary applications with
the library.  If this is what you want to do, use the GNU Lesser General
Public License instead of this License.  But first, please read
\PYGZlt{}https://www.gnu.org/licenses/why\PYGZhy{}not\PYGZhy{}lgpl.html\PYGZgt{}.
\end{sphinxVerbatim}


\section{Pyne License}
\label{\detokenize{LICENSE:pyne-license}}\label{\detokenize{LICENSE:pynelicense}}
\begin{sphinxVerbatim}[commandchars=\\\{\}]
Copyright 2011\PYGZhy{}2020, the PyNE Development Team. All rights reserved.

Redistribution and use in source and binary forms, with or without modification, are
permitted provided that the following conditions are met:

   1. Redistributions of source code must retain the above copyright notice, this list of
      conditions and the following disclaimer.

   2. Redistributions in binary form must reproduce the above copyright notice, this list
      of conditions and the following disclaimer in the documentation and/or other materials
      provided with the distribution.

THIS SOFTWARE IS PROVIDED BY THE PYNE DEVELOPMENT TEAM ``AS IS\PYGZsq{}\PYGZsq{} AND ANY EXPRESS OR IMPLIED
WARRANTIES, INCLUDING, BUT NOT LIMITED TO, THE IMPLIED WARRANTIES OF MERCHANTABILITY AND
FITNESS FOR A PARTICULAR PURPOSE ARE DISCLAIMED. IN NO EVENT SHALL \PYGZlt{}COPYRIGHT HOLDER\PYGZgt{} OR
CONTRIBUTORS BE LIABLE FOR ANY DIRECT, INDIRECT, INCIDENTAL, SPECIAL, EXEMPLARY, OR
CONSEQUENTIAL DAMAGES (INCLUDING, BUT NOT LIMITED TO, PROCUREMENT OF SUBSTITUTE GOODS OR
SERVICES; LOSS OF USE, DATA, OR PROFITS; OR BUSINESS INTERRUPTION) HOWEVER CAUSED AND ON
ANY THEORY OF LIABILITY, WHETHER IN CONTRACT, STRICT LIABILITY, OR TORT (INCLUDING
NEGLIGENCE OR OTHERWISE) ARISING IN ANY WAY OUT OF THE USE OF THIS SOFTWARE, EVEN IF
ADVISED OF THE POSSIBILITY OF SUCH DAMAGE.

The views and conclusions contained in the software and documentation are those of the
authors and should not be interpreted as representing official policies, either expressed
or implied, of the stakeholders of the PyNE project or the employers of PyNE developers.

\PYGZhy{}\PYGZhy{}\PYGZhy{}\PYGZhy{}\PYGZhy{}\PYGZhy{}\PYGZhy{}\PYGZhy{}\PYGZhy{}\PYGZhy{}\PYGZhy{}\PYGZhy{}\PYGZhy{}\PYGZhy{}\PYGZhy{}\PYGZhy{}\PYGZhy{}\PYGZhy{}\PYGZhy{}\PYGZhy{}\PYGZhy{}\PYGZhy{}\PYGZhy{}\PYGZhy{}\PYGZhy{}\PYGZhy{}\PYGZhy{}\PYGZhy{}\PYGZhy{}\PYGZhy{}\PYGZhy{}\PYGZhy{}\PYGZhy{}\PYGZhy{}\PYGZhy{}\PYGZhy{}\PYGZhy{}\PYGZhy{}\PYGZhy{}\PYGZhy{}\PYGZhy{}\PYGZhy{}\PYGZhy{}\PYGZhy{}\PYGZhy{}\PYGZhy{}\PYGZhy{}\PYGZhy{}\PYGZhy{}\PYGZhy{}\PYGZhy{}\PYGZhy{}\PYGZhy{}\PYGZhy{}\PYGZhy{}\PYGZhy{}\PYGZhy{}\PYGZhy{}\PYGZhy{}\PYGZhy{}\PYGZhy{}\PYGZhy{}\PYGZhy{}\PYGZhy{}\PYGZhy{}\PYGZhy{}\PYGZhy{}\PYGZhy{}\PYGZhy{}\PYGZhy{}\PYGZhy{}\PYGZhy{}\PYGZhy{}\PYGZhy{}\PYGZhy{}\PYGZhy{}\PYGZhy{}\PYGZhy{}\PYGZhy{}
The files cpp/measure.cpp and cpp/measure.hpp are covered by:

Copyright 2004 Sandia Corporation.  Under the terms of Contract
DE\PYGZhy{}AC04\PYGZhy{}94AL85000 with Sandia Coroporation, the U.S. Government
retains certain rights in this software.

https://press3.mcs.anl.gov/sigma/moab\PYGZhy{}library

\PYGZhy{}\PYGZhy{}\PYGZhy{}\PYGZhy{}\PYGZhy{}\PYGZhy{}\PYGZhy{}\PYGZhy{}\PYGZhy{}\PYGZhy{}\PYGZhy{}\PYGZhy{}\PYGZhy{}\PYGZhy{}\PYGZhy{}\PYGZhy{}\PYGZhy{}\PYGZhy{}\PYGZhy{}\PYGZhy{}\PYGZhy{}\PYGZhy{}\PYGZhy{}\PYGZhy{}\PYGZhy{}\PYGZhy{}\PYGZhy{}\PYGZhy{}\PYGZhy{}\PYGZhy{}\PYGZhy{}\PYGZhy{}\PYGZhy{}\PYGZhy{}\PYGZhy{}\PYGZhy{}\PYGZhy{}\PYGZhy{}\PYGZhy{}\PYGZhy{}\PYGZhy{}\PYGZhy{}\PYGZhy{}\PYGZhy{}\PYGZhy{}\PYGZhy{}\PYGZhy{}\PYGZhy{}\PYGZhy{}\PYGZhy{}\PYGZhy{}\PYGZhy{}\PYGZhy{}\PYGZhy{}\PYGZhy{}\PYGZhy{}\PYGZhy{}\PYGZhy{}\PYGZhy{}\PYGZhy{}\PYGZhy{}\PYGZhy{}\PYGZhy{}\PYGZhy{}\PYGZhy{}\PYGZhy{}\PYGZhy{}\PYGZhy{}\PYGZhy{}\PYGZhy{}\PYGZhy{}\PYGZhy{}\PYGZhy{}\PYGZhy{}\PYGZhy{}\PYGZhy{}\PYGZhy{}\PYGZhy{}\PYGZhy{}
The files in fortranformat/ are covered by: 

The MIT License. Copyright (c) 2011 Brendan Arnold

Permission is hereby granted, free of charge, to any person obtaining a copy
of this software and associated documentation files (the \PYGZdq{}Software\PYGZdq{}), to deal
in the Software without restriction, including without limitation the rights
to use, copy, modify, merge, publish, distribute, sublicense, and/or sell
copies of the Software, and to permit persons to whom the Software is
furnished to do so, subject to the following conditions:

The above copyright notice and this permission notice shall be included in
all copies or substantial portions of the Software.

THE SOFTWARE IS PROVIDED \PYGZdq{}AS IS\PYGZdq{}, WITHOUT WARRANTY OF ANY KIND, EXPRESS OR
IMPLIED, INCLUDING BUT NOT LIMITED TO THE WARRANTIES OF MERCHANTABILITY,
FITNESS FOR A PARTICULAR PURPOSE AND NONINFRINGEMENT. IN NO EVENT SHALL THE
AUTHORS OR COPYRIGHT HOLDERS BE LIABLE FOR ANY CLAIM, DAMAGES OR OTHER
LIABILITY, WHETHER IN AN ACTION OF CONTRACT, TORT OR OTHERWISE, ARISING FROM,
OUT OF OR IN CONNECTION WITH THE SOFTWARE OR THE USE OR OTHER DEALINGS IN
THE SOFTWARE.

https://bitbucket.org/brendanarnold/py\PYGZhy{}fortranformat/src/
\end{sphinxVerbatim}


\chapter{Contributors}
\label{\detokenize{contributors:contributors}}\label{\detokenize{contributors::doc}}
JADE is the results of a joint effort between \sphinxhref{https://www.niering.it/}{NIER ingegneria},
\sphinxhref{https://ingegneriaindustriale.unibo.it/it}{Università di Bologna (UNIBO)}
and \sphinxhref{https://fusionforenergy.europa.eu/}{Fusion For Energy (F4E)}.

\noindent\sphinxincludegraphics[width=400\sphinxpxdimen]{{nier}.png}

\noindent\sphinxincludegraphics[width=400\sphinxpxdimen]{{unibo}.jpg}

\noindent\sphinxincludegraphics[width=400\sphinxpxdimen]{{f4e}.jpg}

\sphinxstylestrong{Key People:}


\begin{savenotes}\sphinxattablestart
\centering
\begin{tabular}[t]{|\X{50}{200}|\X{50}{200}|\X{50}{200}|\X{50}{200}|}
\hline
\sphinxstyletheadfamily 
Name
&\sphinxstyletheadfamily 
Contribution
&\sphinxstyletheadfamily 
Institution/Company
&\sphinxstyletheadfamily 
Contacts
\\
\hline
Davide Laghi
&
Main developer
&
NIER and UNIBO
&
\sphinxhref{mailto:d.laghi@nier.it}{d.laghi@nier.it}
\\
\hline
Marco Fabbri
&
Project manager and expert
&
F4E
&
\sphinxhref{mailto:marco.fabbri@f4e.europa.eu}{marco.fabbri@f4e.europa.eu}
\\
\hline
Lorenzo Isolan
&
Tester
&
UNIBO
&
\sphinxhref{mailto:lorenzo.isolan2@unibo.it}{lorenzo.isolan2@unibo.it}
\\
\hline
Marco Sumini
&
Expert
&
UNIBO
&
\sphinxhref{mailto:marco.sumini@unibo.it}{marco.sumini@unibo.it}
\\
\hline
\end{tabular}
\par
\sphinxattableend\end{savenotes}


\chapter{List of Publications and Contributions}
\label{\detokenize{publications:list-of-publications-and-contributions}}\label{\detokenize{publications::doc}}

\section{Publications featuring JADE}
\label{\detokenize{publications:publications-featuring-jade}}\begin{itemize}
\item {} 
D. Laghi, M. Fabbri, L. Isolan, R. Pampin, M. Sumini, A. Portone and
A. Trkov, 2020,
“JADE, a new software tool for nuclear fusion data libraries verification \&
validation”, \sphinxstyleemphasis{Fusion Engineering and Design}, \sphinxstylestrong{161} 112075,
doi: \sphinxurl{https://doi.org/10.1016/j.fusengdes.2020.112075}

\item {} 
D. Laghi, M. Fabbri, L. Isolan, M. Sumini, G. Shnabel and A. Trkov, 2021,
“Application Of JADE V\&V Capabilities To The New FENDL v3.2 Beta Release”,
\sphinxstyleemphasis{Nuclear Fusion}, {[}Under minor review{]}

\end{itemize}


\section{Benchmarks Related Publications}
\label{\detokenize{publications:benchmarks-related-publications}}\begin{itemize}
\item {} 
A. Milocco, A. Trkov and I. A. Kodeli, 2010, “The OKTAVIAN TOF experiments in SINBAD: Evaluation of the
experimental uncertainties”, \sphinxstyleemphasis{Annals of Nuclear Energy}, \sphinxstylestrong{37} 443\sphinxhyphen{}449

\item {} 
I.Kodeli, E. Sartori and B. Kirk, “SINBAD \sphinxhyphen{} Shielding Benchmark Experiments \sphinxhyphen{} Status and Planned Activities”,
\sphinxstyleemphasis{Proceedings of the ANS 14th Biennial Topical Meeting of Radiation Protection and Shielding Division},
Carlsbad, New Mexico (April 3\sphinxhyphen{}6, 2006)

\item {} 
D. Leichtle, B. Colling, M. Fabbri, R. Juarez, M. Loughlin,
R. Pampin, E. Polunovskiy, A. Serikov, A. Turner and L. Bertalot, 2018,
“The ITER tokamak neutronics reference model C\sphinxhyphen{}Model”,
\sphinxstyleemphasis{Fusion Engineering and Design}, \sphinxstylestrong{136} 742\sphinxhyphen{}746

\item {} 
M. Sawan, 1994,  “FENDL Neutronics Benchmark: Specifications for the calculational and shielding benchmark”,
(Vienna: INDC(NDS)\sphinxhyphen{}316)

\end{itemize}


\section{Miscellaneous}
\label{\detokenize{publications:miscellaneous}}

\chapter{Introduction}
\label{\detokenize{dev/intro:introduction}}\label{\detokenize{dev/intro::doc}}
TBD


\chapter{Indices and tables}
\label{\detokenize{index:indices-and-tables}}\begin{itemize}
\item {} 
\DUrole{xref,std,std-ref}{genindex}

\item {} 
\DUrole{xref,std,std-ref}{modindex}

\item {} 
\DUrole{xref,std,std-ref}{search}

\end{itemize}



\renewcommand{\indexname}{Index}
\printindex
\end{document}